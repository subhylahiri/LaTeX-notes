% -*- TeX -*- -*- UK -*-
% ----------------------------------------------------------------
% **** -----------------------------------------------------------
\documentclass[11pt]{article}
% Preamble:
\usepackage[margin=1.25in]{geometry}
\usepackage[UKenglish]{babel}
\usepackage[draft,tables]{sl-preamble}
%
% ----------------------------------------------------------------
\vfuzz2pt % Don't report over-full v-boxes if over-edge is small
\hfuzz2pt % Don't report over-full h-boxes if over-edge is small
%\numberwithin{equation}{section}
%\renewcommand{\baselinestretch}{1.5}
% ----------------------------------------------------------------
\crefname{assump}{assumption}{assumptions}
\Crefname{assump}{Assumption}{Assumptions}
\crefalias{enumi}{assump}
% New commands etc.
%
\newcommand{\invd}{^{-\dag}}
\newcommand{\invt}{^{-\mathrm{T}}}
\newcommand{\invc}{^{-\ast}}
\newcommand{\Gammab}{\boldsymbol{\Gamma}}
\newcommand{\Gammap}[1][\mu]{\prn{\Gammab^{#1}}}
\newcommand{\mud}{{\mu\dag}}
\newcommand{\mut}{{\mu\mathrm{T}}}
\newcommand{\muc}{{\mu\ast}}
\renewcommand{\S}{\mathbf{S}}
\newcommand{\T}{\mathbf{T}}
\newcommand{\U}{\mathbf{U}}
\newcommand{\V}{\mathbf{V}}
\newcommand{\Ub}{\overline{\U}}
\newcommand{\Vb}{\overline{\V}}
\newcommand{\A}{\mathbf{A}}
\newcommand{\B}{\mathbf{B}}
\renewcommand{\C}{\mathbf{C}}
\newcommand{\At}{\widetilde{\A}}
\newcommand{\Bt}{\widetilde{\B}}
\newcommand{\Ct}{\widetilde{\C}}
\newcommand{\St}{\widetilde{\S}}
\newcommand{\Cb}{\mathbb{C}}
\newcommand{\sigmab}{\boldsymbol{\sigma}}
\newcommand{\Sigmab}{\boldsymbol{\Sigma}}
\newcommand{\psib}{\boldsymbol{\psi}}
\newcommand{\Psib}{\boldsymbol{\Psi}}
\newcommand{\Lambdab}{\boldsymbol{\Lambda}}
\newcommand{\lambdab}{\boldsymbol{\lambda}}
\newcommand{\Thetab}{\boldsymbol{\Theta}}
\newcommand{\Omegab}{\boldsymbol{\Omega}}
\newcommand{\alphat}{\tilde{\alpha}}
\newcommand{\betat}{\tilde{\beta}}
\newcommand{\chit}{\tilde{\chi}}
%
%================================================================================
\begin{document}
%================================================================================
\title{Spinors in many dimensions}
\author{Subhaneil Lahiri}
\date{}
\maketitle
%================================================================================
\tableofcontents
%================================================================================
\vspace{1em}

We're looking at representations of the $O(p,q)$ Clifford algebra --- a set of matrices $\Gammab^\mu$, $\mu = 1, ..., d$, that satisfy:
%
\begin{equation}\label{eq:clifford}
  \brc{ \Gammab^\mu, \Gammab^\nu } = \eta^{\mu\nu} \I,
\end{equation}
%
where $\brc{ \A, \B } = \A\B + \B\A$ is the anticommutator and $\eta_{\mu\nu}$ is the Lorentz metric with diagonal elements $-1$ for $\mu \leq p$ and $+1$ for $\mu > p$. We also define $q = d - p$.
Euclidean space is \( (p,q) = (0,d) \).
Minkowski space is \( (p,q) = (1,d-1) \) or \( (d-1,1) \) depending on your conventions.

We are going to be very concrete about the representations - when we say ``matrix'' we mean an actual array with numbers in it, not an abstract operator on an abstract vector space.
Really, we are talking about a choice of basis as well as a representation.

Given one representation $\Gammab^\mu$, we can construct another, $\Gammab'^{\mu}$, using a similarity transform with some invertible matrix $\S$:
%
\begin{equation}\label{eq:similarity}
  \Gammab'^{\mu} = \S\inv \Gammab^\mu \S.
\end{equation}
%
This can also be thought of as a change of basis, though not necessarily an orthonormal one.
We can then talk about similarity classes of representations.

We make use of the following statements without justification:
%
\begin{enumerate}
  \item When $d$ is even, there is one similarity class of irreps with dimension $2^{\frac{d}{2}}$.
      \label{it:even}
  \item When $d$ is odd, there are two similarity class of irreps of dimension $2^{\frac{d-1}{2}}$.
      \label{it:odd}
  \item When $d$ is odd, given a representation $\brc{\Gammab^\mu}$, the representation formed by $\brc{ -\Gammab^\mu }$ lies in the other similarity class.
      \label{it:oddchi}
  \item Any matrix that commutes with all of the $\Gammab^\mu$ must be proportional to the identity (otherwise its eigenspaces would be invariant subspaces).
      \label{it:commute}
\end{enumerate}

We use the notation $\floor{x}$ for the floor function: rounding down to the nearest integer.
The dimensionality of the representations is then $2^{\floor{d/2}}$.

\Cref{it:commute} implies that whenever two similarity transforms have the same effect, they must be proportional to each other:
%
\begin{equation}\label{eq:simsim}
  \S\inv \Gammab^\mu \S = \T\inv \Gammab^\mu \T
  \qquad
  \begin{aligned}[t]
     &\implies &
     \prn{\T \S\inv} \Gammab^\mu &= \Gammab^\mu \prn{\T \S\inv}, \\
     &\implies &
     \T \S\inv &= \lambda \I, \\
     &\implies &
     \T &= \lambda \S.
  \end{aligned}
\end{equation}
%
The converse is also clearly true --- two matrices that are proportional to each other produce the same similarity transform.

We can see that the gamma matrices are traceless: from \cref{eq:clifford}: \( \Gammab^{\mu2} = \eta^{\mu\mu} \I = \pm \I \).
Pick some $\nu \neq \mu$, then \( \tr \Gammab^\mu = \eta^{\nu\nu} \tr \Gammab^\mu \Gammab^\nu \Gammab^\nu \).
Cyclicity of the trace \( \implies \tr \Gammab^\mu \Gammab^\nu \Gammab^\nu = \tr \Gammab^\nu \Gammab^\mu \Gammab^\nu \), but \cref{eq:clifford} for $\mu \neq \nu$ means they anticommute \( \implies \tr \Gammab^\mu \Gammab^\nu \Gammab^\nu = - \tr \Gammab^\nu \Gammab^\mu \Gammab^\nu \).
%Therefore \( \tr \Gammab^\mu \Gammab^\nu \Gammab^\nu = 0 \implies \tr \Gammab^\mu = 0 \).

%-------Section------------------------------------------------------------------

\section{Chirality, charge conjugation, \etc matrices}\label{sec:ccmats}


If we have a representation $\brc{\Gammab^\mu}$ we can construct more representations with $\brc{-\Gammab^\mu}$, $\brc{\Gammab^\mud}$, $\brc{-\Gammab^\mud}$, $\brc{\Gammab^\muc}$, $\brc{-\Gammab^\muc}$, $\brc{\Gammab^\mut}$ and $\brc{-\Gammab^\mut}$.
When $d$ is even, \cref{it:even} implies that these representations are similar to the original one.
This means we can find matrices such that
%
\begin{equation}\label{eq:cccmats}
\begin{aligned}
  \Gammab\inv \Gammab^\mu \Gammab &= -\Gammab^\mu, &\quad
  \A  \Gammab^\mu \A\inv  &=  \Gammab^\mud, &\quad
  \At \Gammab^\mu \At\inv &= -\Gammab^\mud, \\ &&
  \B\inv  \Gammab^\mu \B  &=  \Gammab^\muc, &
  \Bt\inv \Gammab^\mu \Bt &= -\Gammab^\muc, \\ &&
  \C  \Gammab^\mu \C\inv  &=  \Gammab^\mut, &
  \Ct \Gammab^\mu \Ct\inv &= -\Gammab^\mut.
\end{aligned}
\end{equation}
%
Some of these matrices have names: $\Gammab$ is called the chirality matrix, either $\C$ or $\Ct$ is usually called the charge conjugation matrix, although we will see in \cref{sec:majorana} that the name would be more appropriate for $\B$ or $\Bt$.
Note that these matrices have only been defined up to an overall scalar multiple, as in \cref{eq:simsim}.
We will fix some of this ambiguity with normalisation conditions below.
We have also defined $\A,\At,\C,\Ct$ in the opposite way to $\Gammab,\B,\Bt$.
This is convenient due to the way that transposing reverses products.

When $d$ is odd, \cref{it:odd,it:oddchi} imply that $\Gammab$ does not exist and exactly one of each pair $(\A, \At)$, $(\B, \Bt)$, $(\C, \Ct)$, will exist.
The statements in the rest of this section should be prefaced with ``assuming that the matrices exist...''.

These transforms are all involutions.
This means we can perform them twice to derive new identities:
%
\begin{equation*}
\begin{gathered}
\begin{alignedat}{5}
  \Gammab^\mu &= -(-\Gammab^\mu) &
        &= -(\Gammab\inv \Gammab^\mu \Gammab) &
        &= \Gammab\inv (-\Gammab^\mu) \Gammab \\ && &&
        &= \Gammab\invsq \Gammab^\mu \Gammab^2 &\quad
  &\implies &
  \Gammab^2 &= \gamma \I, \\
  \Gammab^\mu &= \Gammap[\mud]^\dag &
        &= (\A \Gammab^\mu \A\inv)^\dag &
        &= \A\invd \Gammab^\mud \A^\dag \\ && &&
        &= \A\invd \A \Gammab^\mu \A\inv \A^\dag \\ && &&
        &= (\A\inv \A^\dag)\inv \Gammab^\mu (\A\inv \A^\dag) &
  &\implies &
  \A\inv \A^\dag &= \alpha \I,
\end{alignedat} \\
\begin{aligned}
  \mathclap{\text{similarly:}} &&\qquad
  \B \B^\ast &= \beta \I, &\qquad
  \C\inv \C\trans &= \chi \I, \\
  \At\inv \At^\dag &= \alphat \I, &
  \Bt \Bt^\ast &= \betat \I, &
  \Ct\inv \Ct\trans &= \chit \I.
\end{aligned}
\end{gathered}
\end{equation*}
%
We are free to redefine these matrices as $\Gammab \rightarrow \lambda \Gammab, \A \rightarrow \lambda \A,$ etc.
Under these redefinitions the scalars above transform as
%
\begin{equation*}
  \gamma \rightarrow \lambda^2 \gamma, \qquad
  \alpha \rightarrow \frac{\lambda^\ast}{\lambda} \alpha, \qquad
  \beta \rightarrow \abs{\lambda}^2 \beta, \qquad
  \chi \rightarrow \chi,
\end{equation*}
%
with similar relations for the tilde versions.
These redefinitions allow us to impose some normalisation choices, noting that we can change the phase of $\alpha$ but not its magnitude, vice-versa for $\beta$, and $\chi$ cannot be changed at all.
%
\begin{equation}\label{eq:normscalar}
  \gamma = 1, \qquad
  \phase(\alpha) = 0, \qquad
  \abs{\beta} = 1,
\end{equation}
%
or in terms of the matrices
%
\begin{equation}\label{eq:normmats}
  \Gammab^2 = \I, \qquad
  \A\inv \A^\dag = \A \A\invd, \qquad
  \Psib^\dag \A\inv \A^\dag \Psib \geq 0 \;\; \forall \, \Psib, \qquad
  \B^\ast \B^2 \B^\ast = \I.
\end{equation}
%
Again, the same relations apply to the tilde versions, except that in some situations it will be more convenient to choose to set \( \phase(\alphat) = \pi \).
These choices do not completely eliminate the scalar ambiguity.
For $\Gammab$ we can still use $\lambda = \pm 1$, for $\A$ any $\lambda \in \R$, for $\B$ any $\lambda$ with $\abs{\lambda} = 1$ and for $\C$ any $\lambda \in \Cb$.

The transformations in \cref{eq:cccmats} can also be performed by combining two others in either order.
We can use these combinations to find more scalars and normalisation conditions.
First, for $\A,\B,\C$:
%
\begin{equation*}
\begin{alignedat}{6}
  \Gammab^\mud &= \Gammap[\mut]^\ast &
    &= (\C \Gammab^\mu \C\inv)^\ast &
    &= \C^\ast \Gammab^\muc \C\invc &
    &= \C^\ast \B\inv \Gammab^\mu \B \C\invc &
    &\implies &
  \C^\ast \B\inv &= f \A, \\
    &= \Gammap[\muc]\trans &
    &= (\B\inv \Gammab^\mu \B)\trans &
    &= \B\trans \Gammab^\mut \B\invt &
    &= \B\trans \C \Gammab^\mu \C\inv \B\invt &
    &\implies &
  \B\trans \C &= \tilde{f} \A. \\
\end{alignedat}
\end{equation*}
%
We can relate these scalars to the earlier ones as follows
%
\begin{equation*}
\begin{alignedat}[t]{2}
  \A\inv &= f \, \B \, \C\invc &
    &= \tilde{f} \C\inv \B\invt, \\
  \A^\dag &= \frac{1}{f^\ast} \B\invd \C\trans &
    &= \frac{1}{\tilde{f}^\ast} \C^\dag \B^\ast,
\end{alignedat}
   \quad\implies\quad
\begin{alignedat}[t]{3}
  \alpha \I &= \A\inv \A^\dag &
    &= \frac{\tilde{f}}{f^\ast} \C\inv \B\invt \B\invd \C\trans &
%      &= \frac{\tilde{f}}{f^\ast \beta} \C\inv \C\trans &
      &= \frac{\tilde{f} \chi}{f^\ast \beta} \I, \\ &&
    &= \frac{f}{\tilde{f}^\ast} \B \C\invc \C^\dag \B^\ast &
%      &= \frac{f \chi^\ast}{\tilde{f}^\ast} \B \B^\ast &
      &= \frac{f \beta \chi^\ast}{\tilde{f}^\ast} \I.
\end{alignedat}
\end{equation*}
%
So we have \( \frac{\tilde{f}}{f^\ast} = \frac{\alpha \beta}{\chi} \)
and \( \frac{f}{\tilde{f}^\ast} = \frac{\alpha}{\beta \chi^\ast} \).
If we take the complex-conjugate of the second equation and use the choices in \cref{eq:normscalar},
we find that \( \frac{\tilde{f}}{f^\ast} = \frac{f^\ast}{\tilde{f}} \),
or \( \tilde{f} = \pm f^\ast \) which implies that $\chi = \pm \alpha \beta$.
Furthermore, under the $\C \ra \lambda \C$ ambiguity \( f \rightarrow \lambda^\ast f\),
\(\tilde{f} \rightarrow \lambda \tilde{f} \).
By choosing $\lambda = 1/\tilde{f}$ we set $f = \pm1$ and $\tilde{f} = 1$.
%
\begin{equation}\label{eq:comboABC}
  \A = \B\trans \C = \pm \C^\ast \B\inv, \qquad
  \chi = \pm \alpha \beta.
\end{equation}
%
The first relation is a normalisation choice; the second is forced on us by \cref{eq:normscalar}.
This reduces the scalar ambiguity to simultaneous \( \A \rightarrow r \A \),
\( \B \rightarrow \e^{-\ir\phi} \B \), \( \C \rightarrow r\e^{\ir\phi} \C \)
with \( r, \phi \in \R \).
We can also derive corresponding relations with two of the three matrices/scalars replaced by the tilde versions, but without the freedom to scale $\C$ (we reserve the freedom to rescale $\Ct$ for \cref{eq:chiraltilde}).
%
\begin{equation}\label{eq:ABCtildes}
\begin{aligned}
%  \A &= \B\trans \C = \pm \C^\ast \B\inv, &
  f_1 \A &= \Bt\trans \Ct = \pm \Ct^\ast \Bt\inv, &\quad
  f_2 \At &= \B\trans \Ct = \pm \Ct^\ast \B\inv, &\quad
  f_3 \At &= \Bt\trans \C = \pm \C^\ast \Bt\inv, \\
%  {\chi} &= \pm {\alpha} {\beta}, &
  {\chit} &= \pm {\alpha} {\betat}, &
  {\chit} &= \pm {\alphat} {\beta}, &
  {\chi} &= \pm {\alphat} {\betat}.
\end{aligned}
\end{equation}
%
The signs in each column match.
The first row contains normalisation choices; the second row is forced on us by \cref{eq:normscalar}.
For odd $d$ only one of the four sets of relations from \cref{eq:comboABC,eq:ABCtildes} is possible, and we can choose to set that $f_i = 1$.

When $d$ is even, the same approach leads to relations between tilde versions, tilde-free versions and $\Gammab$.
We find
%
\begin{equation}\label{eq:chiraltilde}
\begin{aligned}
  \A\inv \A^\dag = a \At\inv \At^\dag &= \alpha \I, &\quad
  \At &= \Gammab^\dag \A = a \A \Gammab,            &\quad
  \phase(\alpha) &= 0,                              &\quad
  \abs{a} &= 1, \\
  \B \B^\ast = \pm \Bt \Bt^\ast &= \beta \I,             &
  \Bt &= b \Gammab \B = \pm \frac{1}{b} \B \Gammab^\ast, &
  \phase(b) &= 0,                                        &
  \abs{\beta} &= 1, \\
  \C\inv \C\trans = \pm \Ct\inv \Ct\trans &= \chi \I,    &
  \Ct &= \Gammab\trans \C = \pm \C \Gammab.              &
%  \A = \B\trans \C &= \pm \C^\ast \B\inv, &
%  \chi &= \pm \alpha \beta,
\end{aligned}
\end{equation}
%
The signs across each row match.
The first column is forced on us by other relations; the other columns contain normalisation choices.
The remaining scalar ambiguities are multiplying $\Gammab,\At,\Bt,\Ct$ by $\pm1$
and performing \( \At \rightarrow r \At \), \( \Bt \rightarrow \e^{-\ir\phi} \Bt \),
\( \Ct \rightarrow r\e^{\ir\phi} \Ct \) simultaneously with the transformations of $\A$, $\B$, $\C$ below \cref{eq:comboABC}.

For even $d$ we can derive more relations between these scalars and signs by substituting relations from the second column of \cref{eq:chiraltilde} into \cref{eq:ABCtildes}.
The overall result is
%
\begin{equation}\label{eq:constnorm}
\begin{gathered}
\begin{alignedat}{4}
  \A\inv \A^\dag &= s_1 \At\inv \At^\dag &
        &= \alpha \I, &
    \At &= \Gammab^\dag \A &
        &= s_1 \A \Gammab,
  \\
  \B \B^\ast &= s_2 \Bt \Bt^\ast &
        &= \beta \I, &
    \Bt &= \Gammab \B &
        &= s_2 \B \Gammab^\ast,
  \\
  \C\inv \C\trans &= s_3 \Ct\inv \Ct\trans &
        &= \chi \I, &\qquad
    \Ct &= \Gammab\trans \C &
        &= s_3 \C \Gammab,
  \\
\end{alignedat}
\\
\begin{alignedat}{3}
  \A &= \B\trans \C &
        &= \hat{s}\, \C^\ast \B\inv, &\qquad
    \Gammab^2 &= \I,
  \\
  \A &= \Bt\trans \Ct &
        &= s_1 \hat{s}\, \Ct^\ast \Bt\inv, &\qquad
    \A\inv \A^\dag &= \A \A\invd,
  \\
  s_3 \At &= \B\trans \Ct &
        &= s_2 \hat{s}\, \Ct^\ast \B\inv, &\qquad
    \Psib^\dag \A\inv \A^\dag \Psib &\geq 0 \;\; \forall \, \Psib,
  \\
  s_3 \At &= \Bt\trans \C &
        &= s_3 \hat{s}\, \C^\ast \Bt\inv, &
    \B^\ast \B^2 \B^\ast &= \I,
\end{alignedat}
\\
  \text{or}\quad
  \phase(\alpha) = 0, \qquad
  \abs{\beta} = 1, \qquad \text{and} \quad
  \chi = \hat{s}\, \alpha \beta, \qquad
  s_i^2 = s_1 s_2 s_3 = \hat{s}^2 = 1.
  %s s' s'' &= 1.
\end{gathered}
\end{equation}
%
The first column and last row are forced on us by \cref{eq:normscalar} and other relations; the second column contains normalisation choices.
For odd $d$, on each of the first three rows only one of the two left equations, and none of the right, will exist.
Only one of the left sides of the next four rows will exist, and for that row only one of the two right sides will exist.

In total, we have four undetermined signs $s_i$, $\hat{s}$, and one undetermined complex number $\alpha \beta$.
For odd $d$, we can make any choice for the $s_i$, provided we adjust $\alpha$, $\beta$ and $\hat{s}$ accordingly.

The remaining scalar ambiguities are multiplying $\Gammab,\At,\Bt,\Ct$ by $\pm1$ (for even $d$ only)
and simultaneous \( \A \rightarrow r \A \),
\( \B \rightarrow \e^{-\ir\phi} \B \), \( \C \rightarrow r\e^{\ir\phi} \C \),
\( \At \rightarrow r \At \), \( \Bt \rightarrow \e^{-\ir\phi} \Bt \),
\( \Ct \rightarrow r\e^{\ir\phi} \Ct \) with \( r, \phi \in \R \) (for odd $d$ only half of these will exist).


%-------Section------------------------------------------------------------------

\section{Change of basis}\label{sec:basis}

Suppose we make a change of basis for our spinors, described by a similarity transform:
%
\begin{equation*}
  \Gammab'^{\mu} = \S\inv \Gammab^\mu \S.
\end{equation*}
%
This new representation will have its own set of chirality, charge conjugation, \etc matrices, as in \cref{eq:cccmats}.
How are they related to the corresponding matrices for the previous basis?

Lets illustrate one of these calculations
%
\begin{equation*}
\begin{gathered}
\begin{aligned}
  \Gammab'^\mud
    &= (\S\inv \Gammab^\mu \S)^\dag
    = \S^\dag \Gamma^\mud \S\invd
    = \S^\dag \A \Gammab^\mu \A\inv \S\invd
    = \S^\dag \A \S \Gammab^\mu \S\inv \A\inv \S\invd \\
    &= (\S^\dag \A \S) \Gammab'^\mu (\S^\dag \A \S)\inv,
    \qquad \implies \quad
    \A' \propto \S^\dag \A \S,
\end{aligned}
\\
\begin{aligned}
  \Psib^\dag (\S^\dag \A \S)\inv (\S^\dag \A \S)^\dag \Psib
    &= \Psib^\dag (\S^\dag \A \S)\inv (\S^\dag \A \S)^\dag \Psib
     = \Psib^\dag \S\inv \A\inv \S\invd \S^\dag \A^\dag \S \Psib \\
    &= \Psib^\dag \S\inv \A\inv \A^\dag \S \Psib
     = \alpha\, \Psib^\dag \S\inv \S \Psib
     = \alpha\, \Psib^\dag \Psib
     \geq 0.
\end{aligned}
\end{gathered}
\end{equation*}
%
Once we apply our normalisation condition we find, up to a positive real scalar multiple,
\( \A' = \S^\dag \A \S \).
Similar arguments give us
%
\begin{equation}\label{eq:ccctransform}
\begin{aligned}
  \Gammab' &= \S\inv \Gammab \S, &
  \A' &= \S^\dag \A \S, &
  \B' &= \S\inv \B \S^\ast, &
  \C' &= \S\trans \C \S,
\\ &&
  \At' &= \S^\dag \At \S, &
  \Bt' &= \S\inv \Bt \S^\ast, &
  \Ct' &= \S\trans \Ct \S.
\end{aligned}
\end{equation}
%
We still have the same ambiguities as at the end of the \hyperref[eq:constnorm]{last section}, but it makes sense to choose \( r \e^{\ir\phi} = 1 \).
Regardless, we can substitute \cref{eq:ccctransform} into \cref{eq:constnorm} to compute the scalars:
%
\begin{equation*}
  \alpha' = \alpha, \qquad
  \beta' = \beta, \qquad
  s_i' = s_i, \qquad
  \hat{s}' = \hat{s}.
\end{equation*}
%
In other words, these scalars are invariant under a change of basis and are constant throughout any similarity class of representations.

Although there are two similarity classes of irreps when $d$ is odd, the same set of $\A/\At$, $\B/\Bt$, $\C/\Ct$ will work for both $\brc{\Gammab^\mu}$ and $\brc{ -\Gammab^\mu }$.
Therefore \cref{it:oddchi} implies that both classes have the same $\alpha, \beta, s_i, \hat{s}$.

\Cref{it:even,it:odd} imply that the only thing these constants can depend on is the space-time dimensionality, $p$ and $q$.
In the \hyperref[sec:explicit]{next section} we will compute these quantities for one specific representation that we will construct.
The same constants apply to all representations of that dimensionality.

But before that, how should a spinor transform under this change of basis?
The important thing is that the Dirac equation%
\footnote{The Dirac equation here, and all wave equations below assume the timelike-positive signature, \(p^2 = m^2\).
For the spacelike-positive signature, \(p^2 = - m^2\), multiply \(m\) by \(\ir\).} 
is covariant:
%
\begin{equation*}
  \ir \Gammab^\mu \partial_\mu \Psib = m \Psib
  \qquad \means \qquad
  \ir \Gammab'^\mu \partial_\mu \Psib' = m \Psib'.
\end{equation*}
%
We can construct $\Psib'$ as follows
%
\begin{equation*}
\begin{aligned}
  \ir \Gammab^\mu \partial_\mu \Psib &= m \Psib &
  &\implies &
  \ir \prn{\S \Gammab'^\mu \S\inv} \partial_\mu \Psib &= m \Psib \\ &&
  &\implies &
  \ir \Gammab'^\mu \partial_\mu \prn{\S\inv\Psib} &= m \prn{\S\inv\Psib} &
  &\implies &
  \Psib' &= \S\inv \Psib.
\end{aligned}
\end{equation*}
%
This means we have the following transformations:
%
\begin{equation}\label{eq:spinorbasis}
  \Psib' = \S\inv \Psib, \qquad
  \Psib'^\dag \A' = \Psib^\dag \A \S, \qquad
  \B' \Psib'^\ast = \S\inv \B \Psib^\ast, \qquad
  \Psib'\transp \C' = \Psib\trans \C \S,
\end{equation}
%
and similar for tilde versions.
Note that $\Psib^\dag \Psib$ is not an invariant quantity, as \(\S\) is not necessarily unitary.
However, with the transformations above, we see that $\Psib$ and $\B \Psib^\ast$ are contravariant and $\Psib^\dag\A$ and $\Psib\trans \C$ are covariant.
We can construct these invariants:
%
\begin{equation}\label{eq:spinorinvt}
\begin{gathered}
  \Psib^\dag \A\, \Psib, \qquad
  \Psib^\dag \A\, \B \Psib^\ast, \qquad
  \Psib\trans \C\, \Psib, \qquad
  \Psib\trans \C\, \B \Psib^\ast, \qquad
  \Psib^\dag \A\, \Gammab^\mu \Psib, \\
  \Psib^\dag \A\, \Gammab^\mu \Gammab^\nu \cdots\, \Psib, \qquad
  \Psib^\dag \A\, \Gammab^\mu \Gammab^\nu \cdots\, \B \Psib^\ast, \qquad
  \Psib\trans \C\, \Gammab^\mu \Gammab^\nu \cdots\, \Gammab \Psib, \qquad
  \ldots
\end{gathered}
\end{equation}
%
The quantity \( \Psib^\dag \A \) occurs frequently enough to have its own notation \( \overline{\Psib} \equiv \Psib^\dag \A \).


%-------Section------------------------------------------------------------------

\section{Lorentz transformations}\label{sec:lorentz}

Now we will look at the behaviour of spinors under Lorentz transformations, \(O(p,q)\), that preserve the metric:
%
\begin{equation}\label{eq:lorentz}
  \eta_{\mu\nu}\, \Lambda^\mu{}_\rho\, \Lambda^\nu{}_\sigma = \eta_{\rho\sigma},
  \qquad
  \eta^{\rho\sigma}\, \Lambda^\mu{}_\rho\, \Lambda^\nu{}_\sigma = \eta^{\mu\nu}.
\end{equation}
%
This means that the inverse transforms are given by
%
\begin{equation}\label{eq:invlorentz}
  \prn{\Lambda\inv}^\mu{}_\nu = \eta_{\nu\rho}\, \Lambda^\rho{}_\sigma\, \eta^{\sigma\mu}
  \equiv \Lambda_\nu{}^\mu.
\end{equation}
%

This means that 
%
\begin{equation*}
  \brc{\Lambda^\mu{}_\rho \Gammab^\rho, \Lambda^\nu{}_\sigma \Gammab^\sigma}
    = \brc{\Gammab^\rho, \Gammab^\sigma} \Lambda^\mu{}_\rho\, \Lambda^\nu{}_\sigma
    = 2 \I\, \eta^{\rho\sigma} \Lambda^\mu{}_\rho\, \Lambda^\nu{}_\sigma
    = 2 \I\, \eta^{\mu\nu},
\end{equation*}
%
therefore, either \(\S_\Lambda\) or \(\St_\Lambda\) exists such that
%
\begin{equation}\label{eq:spinlorentz}
  \Lambda^\mu{}_\nu \Gammab^\nu = \S_\Lambda\inv \Gammab^\mu\, \S_\Lambda
  = - \St_\Lambda\inv \Gammab^\mu\, \St_\Lambda.
\end{equation}
%
In even dimensions, both must exist.
In odd dimensions, only one can exist.
We postpone discussion of their relation to Lorentz transformations of spinors to \cref{sec:diraclorentz}.

If we follow the approach in \cref{sec:ccmats} that leads to \cref{eq:constnorm} we can define a set of scalars associated with \(\Lambdab\).
\Cref{eq:constnorm} also implies relations between these constants:
%
\begin{equation*}
\begin{alignedat}{4}
  h_{\Lambda,\Lambda'}^2 &= \frac{g_{\Lambda}g_{\Lambda'}}{g_{\Lambda\Lambda'}}
    \mathrlap{\e^{\ir(\phi_{\Lambda\Lambda'} - \phi_\Lambda - \phi_{\Lambda'})}} &&&
  \Gammab \S_\Lambda
    &= s_\Lambda \S_\Lambda \Gammab 
    \mathrlap{= \hat{g}_\Lambda \St_\Lambda,} \\
  \S_\Lambda \S_{\Lambda'} 
    &= s_{\Lambda} \hat{g}_\Lambda \hat{g}_{\Lambda'}\, \St_\Lambda \St_{\Lambda'} &
    &= h_{\Lambda,\Lambda'}\, \S_{\Lambda\Lambda'}, &\qquad
  \hat{g}_{\Lambda} \St_\Lambda \S_{\Lambda'} 
    &= s_{\Lambda} \hat{g}_{\Lambda'}\, \S_\Lambda \St_{\Lambda'} &
    &= \hat{g}_{\Lambda\Lambda'}h_{\Lambda,\Lambda'}\, \St_{\Lambda\Lambda'}, \\ 
  \S_\Lambda^\dag \A \S_\Lambda
    &= s_1 \abs{\hat{g}_\Lambda}^2 \St_\Lambda^\dag \A \St_\Lambda &
    &= g_\Lambda \A, &
  \S_\Lambda^\dag \At \S_\Lambda
    &= s_1 \abs{\hat{g}_\Lambda}^2 \St_\Lambda^\dag \At \St_\Lambda &
    &= s_\Lambda g_\Lambda \At, \\
  \S_\Lambda\inv \B \S_\Lambda^\ast
    &= \frac{s_2 \hat{g}_\Lambda^\ast}{\hat{g}_\Lambda} \,
      \St_\Lambda\inv \B \St_\Lambda^\ast &
    &= \e^{\ir\phi_\Lambda} \B, &
  \S_\Lambda\inv \Bt \S_\Lambda^\ast
    &= \frac{s_2 \hat{g}_\Lambda^\ast}{\hat{g}_\Lambda} \,
      \St_\Lambda\inv \Bt \St_\Lambda^\ast &
    &= s_\Lambda \e^{\ir\phi_\Lambda} \Bt, \\
  \S_\Lambda\trans \C \S_\Lambda
    &= s_3 \hat{g}_\Lambda^2\, \St_\Lambda\trans \C \St_\Lambda &
    &= g_\Lambda \e^{-\ir\phi_\Lambda} \C, &
  \S_\Lambda\trans \Ct \S_\Lambda 
    &= s_3 \hat{g}_\Lambda^2\, \St_\Lambda\trans \Ct \St_\Lambda &
    &= s_\Lambda g_\Lambda \e^{-\ir\phi_\Lambda} \Ct,
\end{alignedat}
\end{equation*}
%
where \(g_\Lambda\) and \(\phi_\Lambda\) are real, the argument of the square root is positive, \(s_\Lambda = \pm 1\), and \(s_{\Lambda\Lambda'} = s_{\Lambda} s_{\Lambda'}\).
We can use the freedom to scale \(\S_\Lambda\) to set \(g_\Lambda = \e^{\ir\phi_\Lambda} \equiv r_\Lambda = \pm 1\),
and the freedom to scale \(\St_\Lambda\) to set \(\hat{g}_\Lambda = 1\).
This leaves us with:
%
\begin{equation}\label{eq:constnormlorentz}
\begin{alignedat}{4}
  \frac{r_{\Lambda\Lambda'}}{r_{\Lambda} r_{\Lambda'}}
    &= \frac{s_{\Lambda\Lambda'}}{s_{\Lambda} s_{\Lambda'}} 
    = 1, &&&
  \Gammab \S_\Lambda 
    &= s_\Lambda \S_\Lambda \Gammab &
    &= \St_\Lambda, \\
  \S_\Lambda \S_{\Lambda'} 
    &= s_{\Lambda} \St_\Lambda \St_{\Lambda'} &
    &= \S_{\Lambda\Lambda'}, &
  \St_\Lambda \S_{\Lambda'}
    &= s_{\Lambda} \S_\Lambda \St_{\Lambda'} &
    &= \St_{\Lambda\Lambda'}, \\ 
  \S_\Lambda^\dag \A \S_\Lambda 
    &= s_1 \St_\Lambda^\dag \A \St_\Lambda &
    &= r_\Lambda \A, &
  \S_\Lambda^\dag \At \S_\Lambda 
    &= s_1 \St_\Lambda^\dag \At \St_\Lambda &
    &= r_\Lambda s_\Lambda \At, \\
  \S_\Lambda\inv \B \S_\Lambda^\ast 
    &= s_2 \St_\Lambda\inv \B \St_\Lambda^\ast &
    &= r_\Lambda \B, &\qquad
  \S_\Lambda\inv \Bt \S_\Lambda^\ast 
    &= s_2 \St_\Lambda\inv \Bt \St_\Lambda^\ast &
    &= r_\Lambda s_\Lambda \Bt, \\ 
  \S_\Lambda\trans \C \S_\Lambda 
    &= s_3 \St_\Lambda\trans \C \St_\Lambda &
    &= \C, & 
  \S_\Lambda\trans \Ct \S_\Lambda 
    &= s_3 \St_\Lambda\trans \Ct \St_\Lambda &
    &= s_\Lambda \Ct. 
\end{alignedat}
\end{equation}
%
The signs \(r_\Lambda\) and \(s_\Lambda\) must be constant in any connected component of the Lorentz group,
and \(s_\Lambda\) only exists in even dimensions 
(we can make any choice for \(s_\Lambda\)  in odd dimensions).
In odd dimensions, which one of \(\S_\Lambda\) or \(\St_\Lambda\) exists can also only depend on which connected component \(\Lambdab\) is in.
  
%-------Section------------------------------------------------------------------

\subsection{The Lorentz group}\label{sec:group}

If we split \(\Lambda^\mu{}_\nu\) into \((p,q)\) blocks, \cref{eq:lorentz} means that
%
\begin{equation*}
  \Lambdab = 
    \begin{pmatrix}
      \Lambdab_1 & \Lambdab_2 \\
      \Lambdab_3 & \Lambdab_4
    \end{pmatrix}
  \qquad
\begin{aligned}
  \Lambdab_1 \Lambdab_1\trans &= \I_p + \Lambdab_2 \Lambdab_2\trans, &
  \Lambdab_1\trans \Lambdab_1 &= \I_p + \Lambdab_3\trans \Lambdab_3, \\
  \Lambdab_4 \Lambdab_4\trans &= \I_p + \Lambdab_3 \Lambdab_3\trans, &
  \Lambdab_4\trans \Lambdab_4 &= \I_p + \Lambdab_2\trans \Lambdab_2, \\
  \Lambdab_1 \Lambdab_3\trans &= \Lambdab_2 \Lambdab_4\trans, &
  \Lambdab_1\trans \Lambdab_2 &= \Lambdab_3\trans \Lambdab_4.
\end{aligned}
\end{equation*}
%
Considering the eigenvectors of the equations in the first two rows, this means that the SVDs look like
%
\begin{equation*}
\begin{aligned}
  \Lambdab_1  &= \U_\parallel \cosh \Thetab\, \V\trans_\parallel 
              +  \U_\perp \V\trans_\perp, &
  \Lambdab_2  &= \U_\parallel \sinh \Thetab\, \Vb\trans_\parallel, \\
  \Lambdab_3  &= \Ub_\parallel \sinh \Thetab\, \V\trans_\parallel, &
  \Lambdab_4  &= \Ub_\parallel \cosh \Thetab\, \Vb\trans_\parallel 
              +  \Ub_\perp \Vb\trans_\perp,
\end{aligned}
\end{equation*}
%
where \(\Thetab\) is a \(k \times k\) diagonal matrix (\(k = \min(p,q)\)), 
\(\U_\parallel\) and \(\V_\parallel\) are \(p \times k\) submatrices of the \(p \times p\) orthogonal matrices \(\U\) and \(\V\), 
\(\Ub_\parallel\) and \(\Vb_\parallel\) are \(q \times k\) submatrices of the \(q \times q\) orthogonal matrices \(\Ub\) and \(\Vb\), 
and \(\U_\perp\),\(\V_\perp\)/\(\Ub_\perp\),\(\Vb_\perp\) are \(p \times (p-k)\)/\(q \times (q-k)\)  complements of the corresponding submatrices.
One of the pairs \(\U_\perp\),\(\V_\perp\)/\(\Ub_\perp\),\(\Vb_\perp\) will be empty.

If we use \(\cosh\Thetab_{rr}\) to denote  \(\cosh\Thetab\) padded with ones on the diagonal and zeroes elsewhere to make a \(r \times r\) matrix 
and \(\sinh\Thetab_{rs}\) to denote  \(\sinh\Thetab\) padded with zeroes to make a \(r \times s\) matrix,
we can write
%
\begin{equation*}
  \Lambdab = 
    \begin{pmatrix}
      \U\cosh\Thetab_{pp}\V\trans & \U\sinh\Thetab_{pq}\Vb\trans \\
      \Ub\sinh\Thetab_{qp}\V\trans & \Ub\cosh\Thetab_{qq}\Vb\trans
    \end{pmatrix}
  =
    \begin{pmatrix}
      \U & \mathbf{0} \\
      \mathbf{0} & \Ub
    \end{pmatrix}
    \begin{pmatrix}
      \cosh\Thetab_{pp} & \sinh\Thetab_{pq} \\
      \sinh\Thetab_{qp} & \cosh\Thetab_{qq}
    \end{pmatrix}
    \begin{pmatrix}
      \V\trans & \mathbf{0} \\
      \mathbf{0} & \Vb\trans
    \end{pmatrix}.
\end{equation*}
%
This implies that
%
\begin{equation}\label{eq:sgnlorentz}
\begin{lgathered}
    \prn{\det\Lambdab_1}^2 = \prn{\det\Lambdab_4}^2 = \prod_i \cosh^2 \theta_i \geq 1,
  \\
  \sgn \det \Lambdab_1 = \det{\U} \det{\V},
  \qquad
  % \text{where }\; \Theta_{ij} &= \theta_i \delta_{ij} &
  \sgn \det \Lambdab_4 = \det{\Ub} \det{\Vb},
\\
  \det\Lambdab = \det{\U} \det{\Ub} \det{\V} \det{\Vb} = \sgn \det \Lambdab_1 \sgn \det \Lambdab_4,
\end{lgathered}
\end{equation}
%
where we used the fact that the middle matrix in the first expression for \(\Lambdab\) can be permuted%
\footnote{Each row permutation has a corresponding column permutation, so no sign change results.}
to a block-diagonal form with blocks \(
\begin{psmallmatrix}
  \cosh\theta_{i} & \sinh\theta_{i} \\
  \sinh\theta_{i} & \cosh\theta_{i}
\end{psmallmatrix}
\)
and \(\I\).
Note that the determinants of \(\Lambdab_1\) and \(\Lambdab_4\) can never pass through zero, so their signs must be constant in any connected component of \()(p,q)\).

If we send \(\theta_i \to 0\), we see that \(\Lambdab\) is continuously connected to
\(
\begin{psmallmatrix}
  \U\V\trans & \mathbf{0} \\
  \mathbf{0} & \Ub\Vb\trans
\end{psmallmatrix}
\),
which is connected to the identity iff \(\sgn \det \Lambdab_1 = \sgn \det \Lambdab_4 = 1\), 
as \(\U\), etc.\ are connected to any orthogonal matrix with the same determinant.%
\footnote{By multiplying corresponding columns of \(\U\) and \(\V\) by \(-1\) we can change the signs of \(\det\U\) and \(\det\V\) without changing \(\Lambdab\), so it is only the product of determinants that is meaningful.}
Therefore, the group \(O(p,q)\) has four disconnected components described by the signs of the determinants of the diagonal blocks, \(\Lambda_1\) and  \(\Lambda_4\).

The \((+,+)\) component contains all transformations that can be continuously changed to reach the identity.
All matrices in the \((-,+)\) component can be obtained by multiplying a matrix in the \((+,+)\) component by a reflection in one of the first \(p\) dimensions.
Similarly, the \((+,-)\) is the \((+,+)\) component multiplied by a reflection in the last \(q\) dimensions and the \((-,-)\) is the \((+,+)\) component multiplied by one of each type of reflections.

This means we need only consider three types of Lorentz transform: those in the \((+,+)\) component, a reflection in dimension 1, and a reflection in dimension \(p+1\).

The \((+,+)\) component is generated by the Lie algebra \(so(p,q)\):
%
\begin{equation}\label{eq:lielorentz}
  \Lambdab = \exp(\lambdab),
  \qquad
  \eta_{\mu\rho} \lambda^\rho{}_\nu +  \lambda^\rho{}_\mu \eta_{\rho\nu} = 0,
  \quad \text{or} \quad
  \lambda_{\mu\nu} = - \lambda_{\nu\mu}.
\end{equation}
%
We can construct a basis set of generators, \(\mathbf{L}^{ab}\), as
%
\begin{equation}\label{eq:genlorentz}
  L^{ab\mu}{}_\nu = \eta^{a\mu} \delta^b_\nu - \eta^{b\mu} \delta_\nu^a,
  \qquad
  \lambdab = \frac{\lambda_{ab}\, \mathbf{L}^{ab}}{2}.
\end{equation}
%

%-------Section------------------------------------------------------------------

\subsection{Lorentz covariance of the Dirac equation}\label{sec:diraclorentz}

When we use notation like \(\phi(x)\) it is really shorthand for \(\phi(x^1,x^2,\ldots,x^d)\).
We use \(\partial_\mu\phi\) to denote the partial derivative of \(\phi\) with respect to its \(\mu\)'th argument, which may not be \(x^\mu\), e.g.
%
\begin{equation*}
  \pdiff{}{x^\mu} \phi(y(x)) = \pdiff{y^\nu}{x^\mu} \, \partial_\nu \phi(y).
\end{equation*}
%
Under an active Lorentz transform, a scalar field transforms as:
%
\begin{equation*}
\begin{aligned}
  \Lambda: \phi(x) \longmapsto \phi'(x) = \phi(\Lambda\inv x)
  \quad \implies
  \partial_\mu \phi(x) \longmapsto \partial_\mu \phi'(x) 
    &= \pdiff{}{x^\mu} \phi(\Lambda\inv x)
    = \prn{\Lambda\inv}^\nu{}_\mu\, \partial_\nu \phi \\
    &= \Lambda_\mu{}^\nu \partial_\nu \phi, \\
  \partial^\mu \phi(x) \longmapsto \partial^\mu \phi'(x) 
    &\equiv \eta^{\mu\nu} \partial_\nu \phi'
    = \eta^{\mu\nu} \Lambda_\nu{}^\rho\, \partial_\rho \phi \\
    &= \eta^{\mu\nu} \Lambda_\nu{}^\rho \eta_{\rho\sigma}\, \partial^\sigma \phi 
    = \Lambda^\mu{}_\nu \partial^\nu \phi.
\end{aligned}
\end{equation*}
%
We can see that the Klein-Gordon equation%
\footnote{All wave equations below assume the timelike-positive signature, \(p^2 = m^2\).
For the spacelike-positive signature, \(p^2 = - m^2\), multiply \(m\) by \(\ir\).} 
is Lorentz invariant.
Given any particular solution \(\prn{\partial^\mu \partial_\mu + m^2} \phi(x) = 0\):
%
\begin{equation}\label{eq:kleingordoninvt}
  \partial^\mu \partial_\mu \phi' 
    = \Lambda^\mu{}_\rho\, \Lambda_\mu{}^\sigma\, \partial^\rho \partial_\sigma \phi 
    % = \eta^{\mu\nu} \partial_\mu \partial_\nu \phi'
    % = \eta^{\mu\nu} \Lambda_\mu{}^\rho \Lambda_\nu{}^\sigma \partial_\rho \partial_\sigma \phi 
    % = \eta^{\rho\sigma} \partial_\rho \partial_\sigma \phi
    = \delta_\rho^\sigma\, \partial^\rho \partial_\sigma \phi
    = \partial^\rho \partial_\rho \phi
    = -m^2 \phi
    = -m^2 \phi',
\end{equation}
%
and therefore \(\phi'\) is also a solution to the Klein-Gordon equation..

For a vector field, the simple transform for scalars is not enough to satisfy the  vector wave equation.
Given one solution, \(\partial^\mu \partial_\nu V^\nu - \partial^\nu \partial_\nu V^\mu = m^2 V^\mu\):
%
\begin{equation*}
  V'^\mu(x) \stackrel{?}{=} V^\mu(\Lambda\inv x)
  \quad\implies
  \partial^\mu \partial_\nu V'^\nu - \partial^\nu \partial_\nu V'^\mu 
    = \Lambda^\mu{}_\rho \Lambda_\nu{}^\sigma \partial^\rho \partial_\sigma V^\nu
      - \partial^\nu \partial_\nu V^\mu 
    \neq m^2 V'^\mu.
\end{equation*}
%
We also need to transform the spin degrees of freedom, \ie the vector index:
%
\begin{equation}\label{eq:vectorcovt}
\begin{aligned}
  V'^\mu(x) = \Lambda^\mu{}_\nu V^\nu(\Lambda\inv x)
  \quad\implies
  \partial^\mu \partial_\nu V'^\nu - \partial^\nu \partial_\nu V'^\mu 
    &= \Lambda^\mu{}_\rho \prn{\partial^\rho \partial_\nu V^\nu
      - \partial^\nu \partial_\nu V^\rho} \\
    &= m^2 \Lambda^\mu{}_\rho  V^\rho
    = m^2 V'^\mu.
\end{aligned}
\end{equation}
%
Unlike \cref{eq:kleingordoninvt}, the left/right hand sides of the vector wave equation are not invariant under Lorentz transforms.
But they both transform the same way, so \(V^\mu\) being a solution implies that \(V'^\mu\) is also a solution.
In other words, the vector wave equation is covariant under Lorentz transforms.

To have a contravariant Dirac equation we need to find a transformation analogous to \cref{eq:vectorcovt} for spinors.
Given one solution \(\ir \Gammab^\mu \partial_\mu \Psib = m \Psib\):
%
\begin{equation*}
\begin{aligned}
  \Psib'(x) = \S(\Lambda) \Psib(\Lambda\inv x)
  \qquad\implies
  \ir \Gammab^\mu \partial_\mu \Psib'
    &= \ir \Lambda_\mu{}^\nu \Gammab^\mu \S(\Lambda) \partial_\nu \Psib, \\
  m \Psib' &= m \S(\Lambda) \Psib
    = \ir \S(\Lambda) \Gammab^\nu \partial_\nu \Psib,
\end{aligned}
\end{equation*}
%
To have \(\Psib'\) be a solution, we require \(\Lambda_\mu{}^\nu \S\inv(\Lambda) \Gammab^\mu \S(\Lambda) = \Gammab^\nu\).
However, from \cref{eq:spinlorentz} we have
%
\begin{equation*}
  \Lambda_\mu{}^\nu \S\inv_\Lambda \Gammab^\mu \S_\Lambda 
    = \Lambda_\mu{}^\nu \Lambda^\mu{}_\rho \Gammab^\rho
    = \delta^\nu_\rho \Gammab^\rho 
    = \Gammab^\nu.
\end{equation*}
%
Therefore, the Lorentz transform of a spinor is
%
\begin{equation}\label{eq:diraccovt}
\begin{aligned}
  \Lambda: \Psib(x) \longmapsto \Psib'(x) = \S_\Lambda \Psib(\Lambda\inv x)
  \quad \implies
  \ir \Gammab^\mu \partial_\mu \Psib'
    &= \ir \Lambda_\mu{}^\nu \Gammab^\mu \S_\Lambda \partial_\nu \Psib
    = \ir \S_\Lambda \Gammab^\nu \partial_\nu \Psib \\
    &= m \S_\Lambda \Psib
    = m \Psib'.
\end{aligned}
\end{equation}
%
What happens when only \(\St_\Lambda\) exists????????

Note that, according to \cref{eq:constnormlorentz}, we have
%
\begin{equation}\label{eq:cclorentz}
\begin{alignedat}{4}
  \Gammab^\mu \Psib &\longmapsto \Gammab^\mu \Psib' &
    &= \Lambda^\mu{}_\nu \S_\Lambda \Gammab^\nu \Psib, &
  \Gammab \Psib &\longmapsto \Gammab \Psib' &
    &= s_\Lambda \S_\Lambda \Gammab \Psib, \\ 
  \Psib^\dag \A &\longmapsto \Psib'^\dag \A &
    &= r_\Lambda \Psib^\dag \A \S_\Lambda\inv, & \qquad
  \Psib^\dag \At &\longmapsto \Psib'^\dag \At &
    &= r_\Lambda s_\Lambda \Psib^\dag \At \S_\Lambda\inv, \\ 
  \B \Psib^\ast &\longmapsto \B \Psib'^\ast &
    &= r_\Lambda \S_\Lambda \B \Psib, & 
  \Bt \Psib^\ast &\longmapsto \Bt \Psib'^\ast &
    &= r_\Lambda s_\Lambda \S_\Lambda \Bt \Psib, \\ 
  \Psib\trans \C &\longmapsto \Psib'{}\trans \C &
    &= \Psib\trans \C \S_\Lambda\inv, &
  \Psib\trans \Ct &\longmapsto \Psib'{}\trans \Ct &
    &= s_\Lambda \Psib\trans \Ct \S_\Lambda\inv. \\ 
\end{alignedat}
\end{equation}
%
This means that under Lorentz transforms for which \(r_\Lambda = s_\Lambda = 1\), 
\(\overline{\Psib} \Psib\) is a scalar,
\(\overline{\Psib} \Gammab^\mu \Psib\) is a vector,
\(\overline{\Psib} \Gammab^\mu \Gammab^\nu \Psib\) is a second-rank tensor, etc.
We can replace \(\Psib\) with \(\B \Psib^\ast\) or \(\overline{\Psib}\) with \(\Psib\trans \C\) in these expressions. 


%-------Section------------------------------------------------------------------

\subsection{Proper orthochronous Lorentz transforms}\label{sec:properlorentz}

In this section we consider Lorentz transforms with \(\det \Lambdab = \sgn \det \Lambdab_1 \sgn \det \Lambdab_4 = 1\), \ie the \((+,+)\) component that is connected to the identity.
For the identity transform we set \(\S_\Lambda = \I\), which clearly has \(r_\Lambda = s_\Lambda = 1\).
This means that \(\S_\Lambda\) exists and \(r_\Lambda = s_\Lambda = 1\) throughout the \((+,+)\) component.
In even dimensions we can also define \(\St\).

Let us extend \cref{eq:lielorentz,eq:spinlorentz} to spinors:
%
\begin{equation}\label{eq:liespin}
\begin{aligned}
  \Lambdab = \exp(\lambdab),
  \qquad
  \S_\Lambda = \exp(\ir \Sigmab_\lambda)
  \qquad \implies
  \lambda^\mu{}_\nu \Gammab^\nu = \ir \brk{\Gammab^\mu, \Sigmab_\lambda}.
\end{aligned}
\end{equation}
%
If we apply this to the generators in \cref{eq:genlorentz} we find
%
\begin{equation}\label{eq:genspin}
  T^{ab\mu}{}_\nu \Gammab^\nu
    = \eta^{a\mu} \Gammab^b - \eta^{b\mu} \Gammab^a 
    = \ir \brk{\Gammab^\mu, \Sigmab^{ab}},
  \qquad
  \Sigmab^{ab} = \frac{\brk{\Gammab^a, \Gammab^b}}{4\ir},
  \qquad
  \Sigmab_\lambda = \frac{\lambda_{ab} \Sigmab^{ab}}{2}.
\end{equation}
%
The first equation determines \(\Sigmab_{ab}\) up to adding a multiple of the identity, corresponding to the freedom to multiply \(\S\) by a scalar.
In \cref{eq:constnormlorentz} we chose to set \(\det \S_\Lambda = \pm 1\), which requires \(\tr \Sigmab_\lambda = 0\), which fixes the second equation in \cref{eq:genspin} as our chosen solution to the first equation.


%-------Section------------------------------------------------------------------

\subsection{Improper Lorentz transforms}\label{sec:reflectlorentz}

Now we consider a reflection in the \(k\)'th dimension:
%
\begin{equation*}
  \Lambda^\mu{}_\nu 
    = \begin{cases}
        -1 & \mu = \nu = k, \\
        \delta^\mu_\nu & \text{otherwise},
      \end{cases}
  \qquad
  \Gammab^\mu \S_\Lambda
  = \begin{cases}
    -\S_\Lambda \Gammab^\mu & \mu = k, \\
    \S_\Lambda \Gammab^\mu & \text{otherwise},
  \end{cases}
  \qquad
  \Gammab^\mu \St_\Lambda
  = \begin{cases}
    \St_\Lambda \Gammab^\mu & \mu = k, \\
    -\St_\Lambda \Gammab^\mu & \text{otherwise},
  \end{cases}
\end{equation*}
%
We can see that \(\Gammab^k\) satisfies the condition for \(\St_\Lambda\).
If we wish to satisfy the normalisation of \cref{eq:constnormlorentz},
%
\begin{equation}\label{eq:reflect}
  \St_\Lambda = \sqrt{\eta_{kk}s_3}\, \Gammab^k, \qquad
  \S_\Lambda = \sqrt{\eta_{kk}s_3}\, \Gammab \Gammab^k, \qquad
  r_\Lambda = \eta_{kk} s_1, \qquad
  s_\Lambda = -1, 
\end{equation}
%
where the second and fourth do not exist in odd dimensions.

To construct reflections in two different directions, we can multiply the reflections, which also multiplies the signs as in \cref{eq:constnormlorentz}.

In conclusion,
%
\begin{itemize}
  \item In \((+,+)\), \(r_\Lambda = 1\), \(s_\Lambda = 1\), and \(\S\) exists.
  \item In \((-,+)\), \(r_\Lambda = -s_1\), \(s_\Lambda = -1\), and \(\St\) exists.
  \item In \((+,-)\), \(r_\Lambda = s_1\), \(s_\Lambda = -1\), and \(\St\) exists.
  \item In \((-,-)\), \(r_\Lambda = -1\), \(s_\Lambda = 1\), and \(\S\) exists.
\end{itemize}
%
In even dimensions, \(s_\Lambda = \det \Lambdab\). 
In odd dimensions \(\det \Lambdab = 1 \implies \S\) exists, \(\det \Lambdab = -1 \implies \St\) exists.
In any dimensions \(r_\Lambda = \theta(s_1) \sgn \det \Lambdab_1 + \theta(-s_1) \sgn \det \Lambdab_4\).

In \cref{sec:constgendim} we will see that \(s_1 = (-1)^{pq}\).
When \(p\) and \(q\) are both odd \(r_\Lambda = \sgn \det \Lambdab_4\), 
otherwise \(r_\Lambda = \sgn \det \Lambdab_1\).


%-------Section------------------------------------------------------------------

\section{One explicit representation}\label{sec:explicit}

We can construct the Brauer-Weyl representation \cite{brauer1935spinors} of the Clifford algebra in \cref{eq:clifford} in terms of the Pauli sigma-matrices:
%
\begin{equation}\label{eq:paulisigma}
  \sigmab_1 = \begin{pmatrix}
                0 & 1 \\
                1 & 0 \\
              \end{pmatrix}
  , \qquad
  \sigmab_2 = \begin{pmatrix}
                0   & -\ir \\
                \ir & 0    \\
              \end{pmatrix}
  , \qquad
  \sigmab_3 = \begin{pmatrix}
                1 & 0  \\
                0 & -1 \\
              \end{pmatrix}
  , \qquad
  \sigmab_i \sigmab_j = \delta_{ij} \I + \ir \epsilon_{ijk} \sigmab_k.
\end{equation}
%
When $p=0$, we choose
%
\begin{equation}\label{eq:explicit}
\begin{aligned}
  \Gammab^{2a-1} &= \prn{\bigotimes_{k=1}^{a-1} \sigmab_3}
                    \otimes \sigmab_1 \otimes
                    \prn{\bigotimes_{k=1}^{\floor{\frac{d}{2}}-a} \I}, &
    a &= 1, \ldots, \floor{\frac{d}{2}}, \\
  \Gammab^{2a} &= \prn{\bigotimes_{k=1}^{a-1} \sigmab_3}
                  \otimes \sigmab_2 \otimes
                  \prn{\bigotimes_{k=1}^{\floor{\frac{d}{2}}-a} \I}, \\
  \Gammab^d &= \bigotimes_{k=1}^{\floor{d/2}} \sigmab_3, &
    \text{if }
    d &= 1 \pmod 2.
\end{aligned}
\end{equation}
%
For non-zero $p$, we multiply the first $p$ matrices by $-\ir$.
This can be realised as the Hilbert space of $\floorfrac{d}{2}$ spin-$\frac{1}{2}$ systems
\cite{Strathdee:1987jr,strathdee1986extended}.

This representation is convenient because it is systematic and each matrix is either completely real or completely imaginary --- none of them is mixed-complex.
Similarly, each matrix is either Hermitian or antihermitian, and symmetric or antisymmetric.
Finding $\A, \B, \dots$ matrices boils down to finding matrices that commute with all matrices in each category and anticommute with all matrices in the complement.
We can construct candidate matrices from the product of all the Hermitian matrices, the product of all the antihermitian matrices, etc.

When $d$ is even, we can use $\Gammab^{d+1}$ as $\Gammab$, so $\Gammab = \Gammab^\dag = \Gammab^\ast =\Gammab\trans$.
For the others we consider even/odd $p,q$ separately.

%-------Section------------------------------------------------------------------

\subsection{Even \texorpdfstring{$p$ and $q$}{p and q}}\label{sec:eveneven}

We can choose
%
\begin{equation*}
  \A = \mbigg\brk{\bigotimes_{k=1}^{p/2} \sigmab_3}
        \otimes \mbigg\brk{\bigotimes_{k=1}^{q/2} \I},
  \qquad
  \At = \mbigg\brk{\bigotimes_{k=1}^{p/2} \I}
        \otimes \mbigg\brk{\bigotimes_{k=1}^{q/2} \sigmab_3}.
\end{equation*}
%
We can check \cref{eq:constnorm} to see that $\A^\dag = \A$ so $\alpha=1$, $\Gammab \A = \A \Gammab = \At$ so $s_1 = 1$.

We can also choose
%
\begin{equation*}
\begin{aligned}
  p &= 0 \pmod 4: &
  \B &= \mbigg\brk{\bigotimes_{k=1}^{p/4} \prn{\sigmab_2 \otimes \sigmab_1}} \otimes
        \mbigg\brk{\bigotimes_{k=1}^{q/4} \prn{\sigmab_1 \otimes \sigmab_2}}, &
  \beta &= (-1)^{z}, \\ &&
  \Bt &= (\ir)^{2z + \frac{d}{2}}
        \mbigg\brk{\bigotimes_{k=1}^{p/4} \prn{\sigmab_1 \otimes \sigmab_2}} \otimes
        \mbigg\brk{\bigotimes_{k=1}^{q/4} \prn{\sigmab_2 \otimes \sigmab_1}}, &
  \betat &= (-1)^{z+\frac{d}{2}}, \\
  p &= 2 \pmod 4: &
  \B &= (\ir)
        \mbigg\brk{\bigotimes_{k=1}^{p/4} \prn{\sigmab_2 \otimes \sigmab_1}} \otimes
        \mbigg\brk{\bigotimes_{k=1}^{q/4} \prn{\sigmab_2 \otimes \sigmab_1}}, &
  \beta &= (-1)^{z + \frac{d}{2}}, \\ &&
  \Bt &= (-\ir)^{2z + \frac{d-2}{2}}
        \mbigg\brk{\bigotimes_{k=1}^{p/4} \prn{\sigmab_1 \otimes \sigmab_2}} \otimes
        \mbigg\brk{\bigotimes_{k=1}^{q/4} \prn{\sigmab_1 \otimes \sigmab_2}}, &
  \betat &= (-1)^{z}, 
      \end{aligned}
\end{equation*}
%
where we are defining \( \mbig\brk{\bigotimes_{k=1}^{n+\frac{1}{2}} \prn{\mathbf{a} \otimes \mathbf{b}}}
= \brk{\bigotimes_{k=1}^{n} \prn{\mathbf{a} \otimes \mathbf{b}}} \otimes \mathbf{a} \),
and also $z = \floorfrac{p}{4} + \floorfrac{q}{4}$.

Now, according to \cref{eq:constnorm,eq:paulisigma}, $\beta$ is given by the number of $\sigmab_2$'s in $\B$.
We find that \( \beta = (-1)^{z + pd/4} = (-1)^{\floor{(p-q+2)/4}} \) for these values of \(p,q\).
Similarly, we find \( \betat = (-1)^{z + (p+2)d/4} = (-1)^{\floor{(p-q)/4}} \).
We can also check that $\Gammab \B = (-1)^{d/2}\B \Gammab = \Bt$, giving $s_2 = (-1)^{d/2} = (-1)^{(p-q)/2}$.

We can also choose
%
\begin{equation*}
  \C =  \beta \mbigg\brk{\bigotimes_{k=1}^{d/4} \prn{\sigmab_1 \otimes \sigmab_2}},
  \qquad
  \Ct = (\ir)^{\frac{q-p}{2}}
        \mbigg\brk{\bigotimes_{k=1}^{d/4} \prn{\sigmab_2 \otimes \sigmab_1}},
  \qquad
  \chi = (-1)^{\floorfrac{d}{4}},
  \qquad
  \chit = (-1)^{\floorfrac{d+2}{4}}.
\end{equation*}
%
We can check that $\Gammab \C = (-1)^{d/2}\C \Gammab = \Ct$, giving $s_3 = s_2$.
We also find that, from \cref{eq:constnorm}, we have
\( \hat{s} = (-1)^{\floor{d/4}} \beta = (-1)^{p/2} \).
The remaining four identities in \cref{eq:constnorm} can also be verified, leaving us with
%
\begin{equation}\label{eq:consteveneven}
  p, q = 0 \pmod 2 : \quad
  \beta = (-1)^{\floorfrac{p-q+2}{4}}, \quad
  s_2 = s_3 = (-1)^{(p-q)/2}, \quad
  \hat{s} = (-1)^{p/2}.
\end{equation}
%
with unspecified constants equal to 1.

%-------Section------------------------------------------------------------------

\subsection{Odd \texorpdfstring{$p$ and $q$}{p and q}}\label{sec:oddodd}

We choose
%
\begin{equation*}
  \A =  \mbigg\brk{\bigotimes_{k=1}^{\floor{p/2}} \sigmab_3}
        \otimes \sigmab_2
        \otimes \mbigg\brk{\bigotimes_{k=1}^{\floor{q/2}} \sigmab_3},
  \qquad
  \At = (-\ir)
        \mbigg\brk{\bigotimes_{k=1}^{\floor{p/2}} \I}
        \otimes \sigmab_1
        \otimes \mbigg\brk{\bigotimes_{k=1}^{\floor{q/2}} \I}.
\end{equation*}
%
We see again that $\A^\dag = \A$ and $\alpha=1$, and also $\Gammab \A = - \A \Gammab =  \At$ giving us $s_1 = -1$.

We also choose
%
\begin{equation*}
\begin{aligned}
  p &= 1 \pmod 4: &
  \B &= \mbigg\brk{\smashoperator[r]{\bigotimes_{k=1}^{(p-1)/4}} 
        \prn{\sigmab_2 \otimes \sigmab_1}} \otimes
        \sigmab_3 \otimes
        \mbigg\brk{\smashoperator[r]{\bigotimes_{k=1}^{(q-1)/4}}
        \prn{\sigmab_1 \otimes \sigmab_2}}, &
  \beta &= (-1)^z, \\ &&
  \Bt &= (\ir)^{2z + \frac{d-2}{2}}
        \mbigg\brk{\smashoperator[r]{\bigotimes_{k=1}^{(p-1)/4}}
        \prn{\sigmab_1 \otimes \sigmab_2}} \otimes
        \I \otimes
        \mbigg\brk{\smashoperator[r]{\bigotimes_{k=1}^{(q-1)/4}}
        \prn{\sigmab_2 \otimes \sigmab_1}}, &
  \betat &= (-1)^{z + \frac{d-2}{2}}, \\ 
  p &= 3 \pmod 4: &
  \B &= (\ir)
        \mbigg\brk{\smashoperator[r]{\bigotimes_{k=1}^{(p-1)/4}}
        \prn{\sigmab_2 \otimes \sigmab_1}} \otimes
        \I \otimes
        \mbigg\brk{\smashoperator[r]{\bigotimes_{k=1}^{(q-1)/4}}
        \prn{\sigmab_2 \otimes \sigmab_1}}, &
  \beta &= (-1)^{z + \frac{d-2}{2}}, \\ &&
  \Bt &= (-\ir)^{2z + \frac{d+4}{2}}
        \mbigg\brk{\smashoperator[r]{\bigotimes_{k=1}^{(p-1)/4}}
        \prn{\sigmab_1 \otimes \sigmab_2}} \otimes
        \sigmab_3 \otimes
        \mbigg\brk{\smashoperator[r]{\bigotimes_{k=1}^{(q-1)/4}}
        \prn{\sigmab_1 \otimes \sigmab_2}}, &
  \betat &= (-1)^z, 
\end{aligned}
\end{equation*}
%
where $z = \floorfrac{p-1}{4} + \floorfrac{q-1}{4}$.

We find that \( \beta = (-1)^{z + (p-1)(d-2)/4} = (-1)^{\floor{(p-q+2)/4}} \) 
and also \( \betat = (-1)^{z + (p+1)(d-2)/4} = (-1)^{\floor{(p-q)/4}} \) for these values of \(p,q\).
We can also check that $\Gammab \B = -(-1)^{d/2}\B \Gammab = \Bt$, giving $s_2 = (-1)^{(d-2)/2} = (-1)^{(p-q)/2}$.

We can make almost the same choice of $\C$ as for even $p,q$
%
\begin{equation*}
  \C =  (-\ir)^{\frac{d}{2}}
        \mbigg\brk{\bigotimes_{k=1}^{d/4} \prn{\sigmab_1 \otimes \sigmab_2}},
  \qquad
  \Ct = (-1)^{\floorfrac{d}{4}}
        \mbigg\brk{\bigotimes_{k=1}^{d/4} \prn{\sigmab_2 \otimes \sigmab_1}},
  \qquad
  \chi = (-1)^{\floorfrac{d}{4}},
  \quad
  \chit = (-1)^{\floorfrac{d+2}{4}}.
\end{equation*}
%
We can check that $\Gammab \C = (-1)^{d/2}\C \Gammab = \Ct$, giving $s_3 = s_1 s_2$.
We also find that, from \cref{eq:constnorm}, we have
\( \hat{s} = (-1)^{\floor{d/4}} \beta = (-1)^{(q-1)/2} \).
The remaining four identities in \cref{eq:constnorm} can also be verified, leaving us with
%
\begin{equation}\label{eq:constoddodd}
  p, q = 1 \pmod 2 : \quad
  \beta = (-1)^{\floorfrac{p-q+2}{4}}, \quad
  s_1 = -1, \quad
  s_2 = -s_3 = (-1)^{\frac{p-q}{2}}, \quad
  \hat{s} = (-1)^{\frac{q-1}{2}}.
\end{equation}
%
with unspecified constants equal to 1.


%-------Section------------------------------------------------------------------

\subsection{Even \texorpdfstring{$p$}{p}, odd \texorpdfstring{$q$}{q}}\label{sec:evenodd}

We choose
%
\begin{equation*}
  \A =  \mbigg\brk{\bigotimes_{k=1}^{{p/2}} \sigmab_3}
        \otimes \mbigg\brk{\bigotimes_{k=1}^{\floor{q/2}} \sigmab_3},
\end{equation*}
%
The expression for $\At$ in \cref{sec:eveneven} fails to anticommute with $\Gammab^d$.
We see that $\alpha = 1$.
The sign $s_1$ doesn't play a role in \cref{eq:constnorm}, but we can say $s_1 = 1$.

The choices of $\B$ and $\Bt$ in \cref{sec:eveneven} already have the correct (anti)commutation relations with $\Gammab^1, \ldots, \Gammab^{d-1}$.
If we check their behaviour with $\Gammab^d$ we find that $\Gammab^d \B = (-1)^{\floor{d/2}}\B \Gammab^d$ and $\Gammab^d \Bt = (-1)^{\floor{d/2}}\Bt \Gammab^d$, only one of which is the correct behaviour.
We find
%
\begin{equation*}
\begin{aligned}
  p,q &= 0,1 \pmod 4: &
  \B &= \mbigg\brk{\bigotimes_{k=1}^{p/4}
        \prn{\sigmab_2 \otimes \sigmab_1}} \otimes
        \mbigg\brk{\smashoperator[r]{\bigotimes_{k=1}^{(q-1)/4}}
        \prn{\sigmab_1 \otimes \sigmab_2}}, &
  \beta &= (-1)^{z}, \\
  p,q &= 0,3 \pmod 4: &
  \Bt &= \mbigg\brk{\bigotimes_{k=1}^{p/4}
        \prn{\sigmab_1 \otimes \sigmab_2}} \otimes
        \mbigg\brk{\smashoperator[r]{\bigotimes_{k=1}^{(q-1)/4}}
        \prn{\sigmab_2 \otimes \sigmab_1}}, &
  \betat &= (-1)^{z + \frac{d-1}{2}}, \\
  p,q &= 2,3 \pmod 4: &
  \B &= (\ir)
        \mbigg\brk{\bigotimes_{k=1}^{p/4}
        \prn{\sigmab_2 \otimes \sigmab_1}} \otimes
        \mbigg\brk{\smashoperator[r]{\bigotimes_{k=1}^{(q-1)/4}}
        \prn{\sigmab_2 \otimes \sigmab_1}}, &
  \beta &= (-1)^{z + \frac{d-1}{2}}, \\
  p,q &= 2,1 \pmod 4: &
  \Bt &= (-\ir)
        \mbigg\brk{\bigotimes_{k=1}^{p/4}
        \prn{\sigmab_1 \otimes \sigmab_2}} \otimes
        \mbigg\brk{\smashoperator[r]{\bigotimes_{k=1}^{(q-1)/4}}
        \prn{\sigmab_1 \otimes \sigmab_2}}, &
  \betat &= (-1)^{z}, 
\end{aligned}
\end{equation*}
%
where \( z = \floorfrac{p}{4} + \floorfrac{q-1}{4}\).

Because $s_2$ only appears with $\beta$ and $\hat{s}$ in \cref{eq:constnorm}, we can make any choice of $s_2$ with a redefinition of $\beta$ and $\hat{s}$.
We can also check that in all four cases we can write $\beta, \betat$ in the forms $\beta = (-1)^{\floor{(p-q+2)/4}}$ and $\betat = (-1)^{\floor{(p-q)/4}}$, 
which are consistent with \(s_2 = (-1)^{\floorfrac{p-q}{2}}\).
The existence of $\B$ or $\Bt$ is determined by $(d \mod 4) = (p-q \mod 4)$.

Similarly, the choices of $\C$ and $\Ct$ in \cref{sec:eveneven} also have the correct (anti)commutation relations with $\Gammab^1, \ldots, \Gammab^{d-1}$.
If we check their behaviour with $\Gammab^d$ we find that $\Gammab^d \C = (-1)^{\floor{d/2}}\C \Gammab^d$ and $\Gammab^d \Ct = (-1)^{\floor{d/2}} \Ct \Gammab^d$, only one of which is the correct behaviour.
We find
%
\begin{equation*}
\begin{aligned}
  p+q &= 1 = q-p \pmod 4: &
  \C &= \beta \mbigg\brk{\smashoperator[r]{\bigotimes_{k=1}^{(d-1)/4}}
        \prn{\sigmab_1 \otimes \sigmab_2}}, &
  \chi &= (-1)^{\frac{d-1}{4}},
  \\
  p+q &= 3 = q-p \pmod 4: &
  \Ct &= \beta \mbigg\brk{\smashoperator[r]{\bigotimes_{k=1}^{(d-1)/4}}
        \prn{\sigmab_2 \otimes \sigmab_1}}, &
  \chit &= (-1)^{\frac{d+1}{4}}.
\end{aligned}
\end{equation*}
%
Once again, we can choose any value of $s_3$.
We also find that we can write $\chi = (-1)^{\floor{d/4}}$ and $\chit = (-1)^{\floor{(d+2)/4}}$ which is consistent with \(s_3 = s_2\).
This implies that $\hat{s} = (-1)^{p/2}$.
This leaves us with
%
\begin{equation}\label{eq:constevenodd}
  p, q = 0,1 \pmod 2 : \quad
  \beta = (-1)^{\floorfrac{p-q+2}{4}}, \quad
  s_2 = s_3 = (-1)^{\floorfrac{p-q}{2}} \quad
  \hat{s} = (-1)^{\frac{p}{2}}.
\end{equation}
%
with unspecified constants equal to 1.


%-------Section------------------------------------------------------------------

\subsection{Odd \texorpdfstring{$p$}{p}, even \texorpdfstring{$q$}{q}}\label{sec:oddeven}

We choose
%
\begin{equation*}
  \At =
        \mbigg\brk{\bigotimes_{k=1}^{\floor{p/2}} \I}
        \otimes \sigmab_1
        \otimes \mbigg\brk{\bigotimes_{k=1}^{\floor{q/2}} \I}.
\end{equation*}
%
The expression for $\A$ in \cref{sec:oddodd} fails to commute with $\Gammab^d$.
The sign $s_1$ doesn't play a role in \cref{eq:constnorm}, but we can say $s_1 = 1$.
We can then see that $\alpha = 1$.

The choices of $\B$ and $\Bt$ in \cref{sec:oddodd} already have the correct (anti)commutation relations with $\Gammab^1, \ldots, \Gammab^{d-1}$.
If we check their behaviour with $\Gammab^d$ we find that $\Gammab^d \B = -(-1)^{\floor{d/2}}\B \Gammab^d$ and $\Gammab^d \Bt = -(-1)^{\floor{d/2}}\Bt \Gammab^d$, only one of which is the correct behaviour.
We find
%
\begin{equation*}
\begin{aligned}
  p, q &= 1, 2 \pmod 4: &
  \B &= (\ir)
        \mbigg\brk{\smashoperator[r]{\bigotimes_{k=1}^{(p-1)/4}}
        \prn{\sigmab_2 \otimes \sigmab_1}} \otimes
        \sigmab_3 \otimes
        \mbigg\brk{\smashoperator[r]{\bigotimes_{k=1}^{(q-2)/4}}
        \prn{\sigmab_1 \otimes \sigmab_2}}, &
  \beta &= (-1)^{z}, \\
  p, q &= 1, 0 \pmod 4: &
  \Bt &=
        \mbigg\brk{\smashoperator[r]{\bigotimes_{k=1}^{(p-1)/4}}
        \prn{\sigmab_1 \otimes \sigmab_2}} \otimes
        \I \otimes
        \mbigg\brk{\smashoperator[r]{\bigotimes_{k=1}^{(q-2)/4}}
        \prn{\sigmab_2 \otimes \sigmab_1}}, &
  \betat &= (-1)^{z}, \\
  p, q &= 3, 0 \pmod 4: &
  \B &=
        \mbigg\brk{\smashoperator[r]{\bigotimes_{k=1}^{(p-1)/4}}
        \prn{\sigmab_2 \otimes \sigmab_1}} \otimes
        \I \otimes
        \mbigg\brk{\smashoperator[r]{\bigotimes_{k=1}^{(q-2)/4}}
        \prn{\sigmab_2 \otimes \sigmab_1}}, &
  \beta &= (-1)^{z+1}, \\
  p, q &= 3, 2 \pmod 4: &
  \Bt &= (\ir)
        \mbigg\brk{\smashoperator[r]{\bigotimes_{k=1}^{(p-1)/4}}
        \prn{\sigmab_1 \otimes \sigmab_2}} \otimes
        \sigmab_3 \otimes
        \mbigg\brk{\smashoperator[r]{\bigotimes_{k=1}^{(q-2)/4}}
        \prn{\sigmab_1 \otimes \sigmab_2}}, &
  \betat &= (-1)^{z}, \\
\end{aligned}
\end{equation*}
%
where \(z = \floorfrac{p-1}{4} + \floorfrac{q}{4}\).

We can choose any value of $s_2$ for the same reasons as in \cref{sec:evenodd}.
We can also check that in all four cases we can write $\beta, \betat$ in the forms $\beta = (-1)^{\floor{(p-q+2)/4}}$ and $\betat = (-1)^{\floor{(p-q)/4}}$, 
which are consistent with \(s_2 = (-1)^{\floorfrac{p-q}{2}}\).
The existence of $\B$ or $\Bt$ is determined by $(d \mod 4) = (p-q \mod 4)$.

Similarly, the choices of $\C$ and $\Ct$ in \cref{sec:oddodd} also have the correct (anti)commutation relations with $\Gammab^1, \ldots, \Gammab^{d-1}$.
If we check their behaviour with $\Gammab^d$ we find that $\Gammab^d \C = (-1)^{\floor{d/2}}\C \Gammab^d$ and $\Gammab^d \Ct = (-1)^{\floor{d/2}} \Ct \Gammab^d$, only one of which is the correct behaviour.
We find
%
\begin{equation*}
\begin{aligned}
  p+q &= 1 = p-q \pmod 4: &
  \C &= \beta
        \mbigg\brk{\smashoperator[r]{\bigotimes_{k=1}^{(d-1)/4}}
        \prn{\sigmab_1 \otimes \sigmab_2}}, &
  \chi &= (-1)^{\frac{d-1}{4}},
  \\
  p+q &= 3 = p-q \pmod 4: &
  \Ct &= \beta
        \mbigg\brk{\smashoperator[r]{\bigotimes_{k=1}^{(d-1)/4}}
        \prn{\sigmab_2 \otimes \sigmab_1}}, &
  \chit &= (-1)^{\frac{d+1}{4}}.
\end{aligned}
\end{equation*}
%
And again, we can choose any value of $s_3$.
We also find that we can write $\chi = (-1)^{\floor{d/4}}$ and $\chit = (-1)^{\floor{(d+2)/4}}$ which is consistent with \(s_3 = s_2\).
This implies that $\hat{s} = (-1)^{(p-1)/2}$.
This leaves us with
%
\begin{equation}\label{eq:constoddeven}
  p, q = 1,0 \pmod 2 : \quad
  \beta = (-1)^{\floorfrac{p-q+2}{4}}, \quad
  s_2 = s_3 = (-1)^{\floorfrac{p-q}{2}} \quad
  \hat{s} = (-1)^{\frac{p-1}{2}}.
\end{equation}
%
with unspecified constants equal to 1.


%-------Section------------------------------------------------------------------

\subsection{General dimensions}\label{sec:constgendim}

Now let's collect the results of the last four sections,
\cref{eq:consteveneven,eq:constoddodd,eq:constevenodd,eq:constoddeven}.
In all cases we found $\alpha = 1$.
We found that $s_1 = 1$ except when $p$ and $q$ are both odd.
This can be written as $s_1 = (-1)^{pq}$.
We can reconstruct $\alphat = s_1 = (-1)^{pq}$.

The constant $\beta$ could always be expressed as $\beta = (-1)^{\floorfrac{p-q+2}{4}}$,
and $s_2 = (-1)^{\floorfrac{p-q}{2}}$.
In \cref{sec:subtypes} we will see that these two constants, 
$\beta$ and $s_2$, are the most important and both depend only on
\( (p - q \mod 8) \).
We can reconstruct $\betat = s_2 \beta = (-1)^{\floorfrac{p-q}{4}}$.

We found that $s_3 = s_1 s_2$ in all cases, which is equivalent to $s_3 = (-1)^{\floorfrac{d}{2}}$.
The sign $\hat{s}$ took different forms depending on $p,q \mod 2$.
Rather than try to combine them into one formula, \(\hat{s} = (-1)^{(q-1)\floorfrac{p}{2} + q \floorfrac{q}{2}}\),
but it is easier to reconstruct $\chi = (-1)^{\floor{d/4}}$ and $\chit = s_3 \chi = (-1)^{\floor{(d+2)/4}}$.

These results are summarised in \cref{tab:exist}.


\begin{table}
  \centering
  \begin{tabular}{CAACAA}
  \toprule
    \text{Matrix}  & \colhead{Similarity} & \colhead{Nonexistent when\ldots} & \text{Doubling}
                    & \colhead{Constant}                 & \colhead{Sign} \\
  \midrule
    \Gammab & -&\Gammab^\mu        & p\pm q &= 1 \pmod 2              & \Gammab^2
                    & \gamma &= 1                        &                           \\
    \A      &  &\Gammab^\mud       & p,q &= 1,0 \pmod 2               & \A\inv \A^\dag
                    & \alpha &= 1                        &                           \\
    \At     & -&\Gammab^\mud       & p,q &= 0,1 \pmod 2               & \At\inv \At^\dag
                    & \alphat &= (-1)^{pq}               & s_1 &= (-1)^{pq}     \\
    \B      &  &\Gammab^\muc       & p-q &= 1 \pmod 4                 & \B \B^\ast
                    & \beta &= (-1)^{\floorfrac{p-q+2}{4}}  &                           \\
    \Bt     & -&\Gammab^\muc       & p-q &= 3 \pmod 4                 & \Bt \Bt^\ast
                    & \betat &= (-1)^{\floorfrac{p-q}{4}} & s_2 &= (-1)^{\floorfrac{p-q}{2}} \\
    \C      &  &\Gammab^\mut       & p+q &= 3 \pmod 4                 & \C\inv \C\trans
                    & \chi &= (-1)^{\floorfrac{p+q}{4}}     & \hat{s} &= \beta\chi      \\
    \Ct     & -&\Gammab^\mut       & p+q &= 1 \pmod 4                 & \Ct\inv \Ct\trans
                    & \chit &= (-1)^{\floorfrac{p+q+2}{4}}  & s_3 &= (-1)^{\floorfrac{p+q}{2}}  \\
  \bottomrule
  \end{tabular}
  \caption{Existence of chirality, charge conjugation, \etc matrices and their associated constants.
  The second column lists the result of the matrix in the first column being used in a similarity transform on $\Gammab^\mu$.
  The fourth column lists the matrix that performs the operation from the second column twice.
  The result must be proportional to the identity matrix, with constant of proportionality from the fifth column.
  The sixth column contains additional signs defined in \cref{eq:constnorm}.
  }\label{tab:exist}
\end{table}


%-------Section------------------------------------------------------------------

\section{Majorana, Weyl and Majorana-Weyl spinors}\label{sec:subtypes}

In general, the spinors that we have constructed are known as Dirac spinors.
Depending on the dimensionality, it can be useful to split Dirac spinors into some subspaces.


\begin{table}
  \centering
  \begin{tabular}{ccccl}
  \toprule
    % after \\: \hline or \cline{col1-col2} \cline{col3-col4} ...
    $p-q \mod 8$ & $\beta$ & $\betat$ & $s_2$ & Spinor types  \\
  \midrule
    0 & 1  & 1  & 1  & Majorana-Weyl            \\
    1 & -  & 1  & -  & Majorana                 \\
    2 & -1 & 1  & -1 & Majorana, Weyl           \\
    3 & -1 & -  & -  & Symplectic-Majorana      \\
    4 & -1 & -1 & 1  & Symplectic-Majorana-Weyl \\
    5 & -  & -1 & -  & Symplectic-Majorana      \\
    6 & 1  & -1 & -1 & Majorana, Weyl           \\
    7 & 1  & -  & -  & Majorana                 \\
  \bottomrule
  \end{tabular}
  \caption{Minimal spinor types in various dimensions.
  Entries ``-'' indicate that the relevant matrices do not exist.
  ``Majorana, Weyl'' indicates that Majorana spinors exist, Weyl spinors exist, but Majorana-Weyl do not.
  If Majorana-Weyl spinors exist, then Majorana spinors and Weyl spinors also exist.
  Similarly, if Symplectic-Majorana-Weyl spinors exist, then Symplectic-Majorana spinors and Weyl spinors also exist.
  Dirac spinors always exist.
  }\label{tab:types}
\end{table}


%-------Section------------------------------------------------------------------

\subsection{Weyl spinors}\label{sec:weyl}

In even-dimensional spaces, we have $\Gammab^2 = \I$, so its eigenvalues are $\pm 1$.
We can argue that $\tr \Gammab = 0$ using the same reasoning as for the $\Gammab^\mu$.
Therefore, the two eigenspaces have the same dimensionality, $2^{\floor{d/2}-1}$.
Members of the +1 eigenspace are called \emph{Weyl} or \emph{chiral spinors}, the -1 eigenspace are antichiral spinors.

Is this a Lorentz invariant concept?
Note that from \cref{eq:constnormlorentz,sec:reflectlorentz} we have \(\Gammab \S_\Lambda \Psib = \det (\Lambda)\, \S_\Lambda \Gammab \Psib\).
Therefore, \emph{proper} Lorentz transforms map chiral spinors to chiral spinors and antichiral to antichiral.
However, \emph{improper} Lorentz transforms swap chiral and antichiral spinors.

Any Dirac spinor can be written as the sum of its chiral and anti-chiral parts:
%
\begin{equation}\label{eq:Weylanti}
  \Psib_R \equiv \brkfrac{\I+\Gammab}{2} \Psib,
  \quad
  \Psib_L \equiv \brkfrac{\I-\Gammab}{2} \Psib,
  \qquad
  \Gammab \Psib_R = \Psib_R,
  \quad
  \Gammab \Psib_L = - \Psib_L.
  \quad
  \Psib = \Psib_R + \Psib_L,
\end{equation}
%
These two subspaces are invariant under the changes of basis considered in \cref{sec:basis}.
We can also talk of a \emph{Weyl basis} for spinors (a distinct concept from Weyl spinors), which is a basis in which the chirality matrix takes the block form
%
\begin{equation}\label{eq:weylbasis}
  \Gammab = \begin{pmatrix}
              \I         & \mathbf{0} \\
              \mathbf{0} & -\I        \\
            \end{pmatrix}
  \qquad
  \implies
  \qquad
  \Gammab^\mu = \begin{pmatrix}
                  \mathbf{0} & \sigmab^\mu \\
                  \overline{\sigmab}^\mu & \mathbf{0} \\
                \end{pmatrix}
  ,
  \qquad
  \Psib = \begin{pmatrix}
            \psib_R \\
            \psib_L \\
          \end{pmatrix}
  .
\end{equation}
%
In such a basis the chiral part of a spinor is its upper half and the antichiral part its lower half.
From \cref{eq:cccmats} only the off-diagonal blocks of the $\Gammab^\mu$ may be non-zero.

Are there any chiral solutions to the Dirac equation?
It would be necessary for both $\Psib$ and $\Gammab \Psib$ to be solutions:
%
\begin{equation*}
  \ir \Gammab^\mu \partial_\mu \Psib = m \Psib
  \quad \text{and} \quad
  \ir \Gammab^\mu \partial_\mu (\Gammab \Psib) = m (\Gammab \Psib),
  \qquad \text{but} \quad
\begin{aligned}[t]
  \ir \Gammab^\mu \partial_\mu (\Gammab \Psib)
    &= - \Gammab (\ir \Gammab^\mu \partial_\mu \Psib) \\
    &= - m (\Gammab \Psib).
\end{aligned}
\end{equation*}
%
Therefore this is only possible when $m=0$.
Particles with mass can still be discussed in the language of Weyl spinors, they just have to come in a chiral-antichiral pair.%
\footnote{They could still satisfy a Majorana condition, as discussed in the next three sections}
A \emph{single} Weyl spinor can only be used for massless particles.

%-------Section------------------------------------------------------------------

\subsection{Majorana spinors}\label{sec:majorana}

Can we impose a reality condition, like $\Psib^\ast = \Psib$?
Changes of basis can be complex, so a spinor that is real in one basis will not necessarily be real in another basis.
The problem comes from the fact that $\Psib$ and $\Psib^\ast$ transform differently under a change of basis.
If we want an invariant characterisation of real spinors, we need a version of complex conjugation that transforms the same way.
From \cref{eq:spinorbasis} we see that this is provided by \emph{charge conjugation}: \( \Psib \to \B \Psib^\ast \) or \( \Bt \Psib^\ast \).
We will focus on $\B$ below, but our discussion applies equally well to $\Bt$.

But are there any self-conjugate spinors, \ie solutions to the condition \( \Psib = \B \Psib^\ast \)?%
\footnote{Many authors write this condition as \( \overline{\Psib} = \Psib\trans \C \).}
Let's perform a consistency  check by substituting thus equation into itself
%
\begin{equation*}
  \Psib = \B \Psib^\ast = \B (\B \psi^\ast)^\ast = \B \B^\ast \Psib = \beta \Psib.
\end{equation*}
%
Therefore, unless $\beta = 1$ there are no self-conjugate solutions
(you might try to get around it by using something like \( \Psib = \ir \B \Psib^\ast \) but, because charge conjugation is antilinear, it doesn't work).
Assuming either $\beta$ or $\beta$ is one, do any solutions exist?
To answer, we can use a doubling trick where we map $\Cb^n \to \R^{2n}$.
In block matrix format:
%
\begin{equation*}
\begin{gathered}
  \Psib \to \begin{pmatrix}
              \Re \Psib \\
              \Im \Psib \\
            \end{pmatrix}
  \equiv    \begin{pmatrix}
              \psib_r \\
              \psib_i \\
            \end{pmatrix}
  , \qquad
  \B \to \begin{pmatrix}
           \Re \B & -\Im \B \\
           \Im \B &  \Re \B \\
         \end{pmatrix}
  \equiv \begin{pmatrix}
           \B_r & -\B_i \\
           \B_i &  \B_r \\
         \end{pmatrix}
  , \\
%\begin{aligned}
  \implies
  \B \B^\ast \to
         \begin{pmatrix}
           \B_r & -\B_i \\
           \B_i &  \B_r \\
         \end{pmatrix}
         \begin{pmatrix}
            \B_r & \B_i \\
           -\B_i & \B_r \\
         \end{pmatrix}
  =
         \begin{pmatrix}
           \B_r^2 + \B_i^2 & [\B_r, \B_i]    \\
           [\B_i, \B_r]    & \B_r^2 + \B_i^2 \\
         \end{pmatrix}
  =
         \begin{pmatrix}
           \beta \I   & \mathbf{0} \\
           \mathbf{0} & \beta \I   \\
         \end{pmatrix}
  .
%\end{aligned}
\end{gathered}
\end{equation*}
%
Multiplication of these real doubled matrices is equivalent to multiplication of the original complex matrices.
In this language complex conjugation is a linear operation, so
%
\begin{equation*}
  \B \Psib^\ast \to
         \begin{pmatrix}
           \B_r & -\B_i \\
           \B_i &  \B_r \\
         \end{pmatrix}
         \begin{pmatrix}
           \I  & \mathbf{0} \\
           \mathbf{0} & -\I   \\
         \end{pmatrix}
         \begin{pmatrix}
           \psib_r \\
           \psib_i \\
         \end{pmatrix}
  =
         \begin{pmatrix}
           \B_r &  \B_i \\
           \B_i & -\B_r \\
         \end{pmatrix}
         \begin{pmatrix}
           \psib_r \\
           \psib_i \\
         \end{pmatrix}
  \equiv \mathcal{B} \psib.
\end{equation*}
%
Therefore, it comes down to the eigenvalues of the operator in the final expression.
But
%
\begin{equation*}
  \mathcal{B}^2 =
    \begin{pmatrix}
      \B_r &  \B_i \\
      \B_i & -\B_r \\
    \end{pmatrix}^2
  =
    \begin{pmatrix}
      \B_r^2 + \B_i^2 & [\B_r, \B_i]    \\
      [\B_i, \B_r]    & \B_r^2 + \B_i^2 \\
    \end{pmatrix}
  =
    \begin{pmatrix}
      \beta \I   & \mathbf{0} \\
      \mathbf{0} & \beta \I   \\
    \end{pmatrix}.
\end{equation*}
%
Therefore, when $\beta=1$ the eigenvalues must be $\pm 1$.
Furthermore, the operator has zero trace, so both eigenspaces have half of the dimensionality of the space.
If $\beta=-1$ but $\betat=1$, we can use $\Bt$ instead.
Consulting \cref{tab:types}, we can use $\B$ when $p-q = 0,6,7 \pmod 8$ and we can use $\Bt$ when $p-q = 0,1,2 \pmod 8$.

Either way, Dirac spinors can be split into their real and imaginary parts
%
\begin{equation}\label{eq:majorana}
  \psib_r = \frac{\Psib + \B \Psib^\ast}{2},
  \quad
  \psib_i = \frac{\Psib - \B \Psib^\ast}{2\ir},
  \qquad
  \psib_r = \B \psib^\ast_r,
  \quad
  \psib_i = \B \psib^\ast_i,
  \quad
  \Psib = \psib_r + \ir \psib_i.
\end{equation}
%
Here both $\psib_r$ and $\psib_i$ are \emph{Majorana spinors}.
Dirac spinors have $2^{\floor{d/2}}$ \emph{complex dimensions}, so Majorana spinors have $2^{\floor{d/2}}$ \emph{real dimensions}.
Note that anti-Majorana spinors are just Majorana spinors multiplied by $\ir$, so we don't need to think of them as another type of spinor.
We can also talk of a Majorana basis for spinors (again, a distinct concept from Majorana spinors) where either $\B$ (or $\Bt$) is the identity matrix.
In such a basis all of the $\Gammab^\mu$ are real (or all imaginary), and the condition for Majorana spinors is simply that they are real.

%-------Section------------------------------------------------------------------

\subsection{Symplectic-Majorana spinors}\label{sec:symplecticmajorana}

If we consult \cref{tab:types}, we see that when \( p-q = 3,4,5 \pmod 8 \), neither $\beta$ nor $\betat$ is one.
In these cases it is impossible for a spinor to be self-conjugate
(the doubling trick doesn't help because the eigenvalues are $\pm\ir$, which are matrices in that language, not scalars).
However, it is possible to group spinors into pairs such that
%
\begin{equation}\label{eq:pseudomajorana}
  \B \psib_1^\ast = \psib_2,
  \qquad
  \B \psib_2^\ast = -\psib_1.
\end{equation}
%
This endows spinors with a symplectic structure, like phase space in Hamiltonian mechanics.
%We can define a matrix $\Omegab$ such that $\Omegab \psib_1 = \psib_2$ and $\Omegab \psib_2 = -\psib_1$, and therefore $\Omegab^2 = - \I$.
%Then, \cref{eq:pseudomajorana} reads
%%
%\begin{equation}\label{eq:symplecticmajorana}
%  \B \psib^\ast = \Omegab \psib.
%\end{equation}
%%
%Consistency of this condition requires
%%
%\begin{equation*}
%\begin{aligned}
%  - \Omegab \B \psib^\ast &= \psib, \\
%  \Omegab^\ast \B^\ast \psib &= -\psib^\ast, \\
%  \Omegab \B \Omegab^\ast \B^\ast \psib &= - \Omegab \B \psib^\ast = \psib.
%\end{aligned}
%\end{equation*}
%%
%So we require one to be an eigenvalue of $\Omegab \B \Omegab^\ast \B^\ast$.
%If we wish to split spinors into their real and imaginary parts, we require \( \Omegab \B \Omegab^\ast \B^\ast = \I \).
%If we pre-multiply by $\Omegab$ and post-multiply by $\B$, we find the second of the two required properties of an $\Omegab$:
%%
%\begin{equation}\label{eq:symplectic}
%  \Omegab^2 = - \I,
%  \qquad
%  \Omegab \B = \B \Omegab^\ast.
%\end{equation}
%%
%This means that under a change of basis it transforms as $\Omegab' = \S\inv \Omegab \S$.
%If we find a suitable matrix in one representation (albeit not unique), we can find one in all others.
%For the Brauer-Weyl representation, when $\beta/\betat = -1$, $\ir\B/\ir\Bt$ has the required properties.
%It will take a different form in other representations, as they transform differently under basis changes, given by $\Omegab = \ir \B \S^\ast \S\inv$ where $\S$  generates a change of basis to the Brauer-Weyl representation.
%
%We can split Dirac spinors into their ``real'' and ``imaginary'' parts
%%
%\begin{equation}\label{eq:psedoreim}
%  \psib_r = \frac{\psib - \Omega \B \psib^\ast}{2},
%  \quad
%  \psib_i = \frac{\psib + \Omega \B \psib^\ast}{2\ir},
%  \qquad
%  \B \psib_r^\ast = \Omegab \psib_r,
%  \quad
%  \B \psib_i^\ast = \Omegab \psib_i,
%  \quad
%  \psib = \psib_r + \ir \psib_i.
%\end{equation}
%%
%Here both $\psib_r$ and $\psib_i$ are \emph{symplectic-Majorana}, \emph{pseudo-Majorana} or \emph{pseudo-real spinors}.
%Symplectic-Majorana spinors have $2^{\floor{d/2}}$ real dimensions.
%Note that, as with Majorana spinors, symplectic-anti-Majorana spinors are just symplectic-Majorana spinors multiplied by $\ir$, so we don't need to think of them as another type of spinor either.


%================================================================================
\bibliographystyle{utcaps_nat}
\bibliography{journals,qft,gr,maths}
%================================================================================
\end{document}

