% -*- TeX -*- -*- UK -*-
% ----------------------------------------------------------------
% !TEX program = pdflatex
% !BIB program = bibtex
% **** -----------------------------------------------------------
\documentclass[11pt]{article}
% Preamble:
\usepackage[margin=1.25in]{geometry}
\usepackage[UKenglish]{babel}
\usepackage[draft,tables]{sl-preamble}
%
% ----------------------------------------------------------------
\vfuzz2pt % Don't report over-full v-boxes if over-edge is small
\hfuzz2pt % Don't report over-full h-boxes if over-edge is small
%\numberwithin{equation}{section}
%\renewcommand{\baselinestretch}{1.5}
% ----------------------------------------------------------------
\crefname{assump}{assumption}{assumptions}
\Crefname{assump}{Assumption}{Assumptions}
\crefalias{enumi}{assump}
% New commands etc.
%
\newcommand{\invd}{^{-\dag}}
\newcommand{\invt}{^{-\mathrm{T}}}
\newcommand{\invc}{^{-\ast}}
\newcommand{\Gammab}{\boldsymbol{\Gamma}}
\newcommand{\Gammap}[1][\mu]{\prn{\Gammab^{#1}}}
\newcommand{\mud}{{\mu\dag}}
\newcommand{\mut}{{\mu\mathrm{T}}}
\newcommand{\muc}{{\mu\ast}}
\renewcommand{\S}{\mathbf{S}}
\newcommand{\T}{\mathbf{T}}
\newcommand{\A}{\mathbf{A}}
\newcommand{\B}{\mathbf{B}}
\renewcommand{\C}{\mathbf{C}}
\newcommand{\At}{\widetilde{\mathbf{A}}}
\newcommand{\Bt}{\widetilde{\mathbf{B}}}
\newcommand{\Ct}{\widetilde{\mathbf{C}}}
\newcommand{\Cb}{\mathbb{C}}
\newcommand{\sigmab}{\boldsymbol{\sigma}}
\newcommand{\psib}{\boldsymbol{\psi}}
\newcommand{\Omegab}{\boldsymbol{\Omega}}
\newcommand{\alphat}{\tilde{\alpha}}
\newcommand{\betat}{\tilde{\beta}}
\newcommand{\chit}{\tilde{\chi}}
%
%================================================================================
\begin{document}
%================================================================================
\title{Spinors in many dimensions}
\author{Subhaneil Lahiri}
\date{}
\maketitle
%================================================================================

We're looking at representations of the $SO(p,q)$ Clifford algebra --- a set of matrices $\Gammab^\mu$, $\mu = 1, ..., d$, that satisfy:
%
\begin{equation}\label{eq:clifford}
  \brc{ \Gammab^\mu, \Gammab^\nu } = \eta^{\mu\nu} \I,
\end{equation}
%
where $\brc{ \A, \B } = \A\B + \B\A$ is the anticommutator and $\eta_{\mu\nu}$ is the Lorentz metric with diagonal elements $+1$ for $\mu \leq p$ and $-1$ for $\mu > p$. We also define $q = d - p$.
Euclidean space is \( (p,q) = (0,d) \).
Minkowski space is \( (p,q) = (1,d-1) \) or \( (d-1,1) \).

Given one representation $\Gammab^\mu$, we can construct another, $\Gammab'^{\mu}$, using a similarity transform with some invertible matrix $\S$:
%
\begin{equation}\label{eq:similarity}
  \Gammab'^{\mu} = \S\inv \Gammab^\mu \S.
\end{equation}
%
We can then talk about similarity classes of representations.

We make use of the following statements without justification:
%
\begin{enumerate}
  \item When $d$ is even, there is one similarity class of irreps with dimension $2^{ \frac{d}{2} }$. \label{it:even}
  \item When $d$ is odd, there are two similarity class of irreps of dimension $2^{ \frac{d-1}{2} }$. \label{it:odd}
  \item When $d$ is odd, given a representation $\brc{\Gammab^\mu}$, the representation formed by $\brc{ -\Gammab^\mu }$ lies in the other similarity class. \label{it:oddchi}
  \item Any matrix that commutes with all of the $\Gammab^\mu$ must be proportional to the identity (otherwise its eigenspaces would be invariant subspaces). \label{it:commute}
\end{enumerate}

We use the notation $\floor{x}$ for the floor function: rounding down to the nearest integer.
The dimensionality of the representations is then $2^{\floor{d/2}}$.

\Cref{it:commute} implies that whenever two similarity transforms have the same effect, they must be proportional to each other:
%
\begin{equation}\label{eq:simsim}
  \S\inv \Gammab^\mu \S = \T\inv \Gammab^\mu \T
  \qquad
  \begin{aligned}[t]
     &\implies &
     \prn{\T \S\inv} \Gammab^\mu &= \Gammab^\mu \prn{\T \S\inv}, \\
     &\implies &
     \T \S\inv &= \lambda \I, \\
     &\implies &
     \T &= \lambda \S.
  \end{aligned}
\end{equation}
%
The converse is also clearly true --- two matrices that are proportional to each other produce the same similarity transform.

We can see that the gamma matrices are traceless: from \cref{eq:clifford}: \( \Gammab^{\mu2} = \eta^{\mu\mu} \I = \pm \I \).
Pick some $\nu \neq \mu$, then \( \tr \Gammab^\mu = \eta^{\nu2} \tr \Gammab^\mu \Gammab^\nu \Gammab^\nu \).
Cyclicity of the trace \( \implies \tr \Gammab^\mu \Gammab^\nu \Gammab^\nu = \tr \Gammab^\nu \Gammab^\mu \Gammab^\nu \), but \cref{eq:clifford} for $\mu \neq \nu \implies$ anti-commutation, so \( \tr \Gammab^\mu \Gammab^\nu \Gammab^\nu = - \tr \Gammab^\nu \Gammab^\mu \Gammab^\nu \).


%-------Section------------------------------------------------------------------

\section{Chirality, charge conjugation, \etc matrices}\label{sec:ccmats}


If we have a representation $\brc{\Gammab^\mu}$ we can construct more representations with $\brc{-\Gammab^\mu}$, $\brc{\Gammab^\mud}$, $\brc{-\Gammab^\mud}$, $\brc{\Gammab^\muc}$, $\brc{-\Gammab^\muc}$, $\brc{\Gammab^\mut}$ and $\brc{-\Gammab^\mut}$.
When $d$ is even, \cref{it:even} implies that these representations are similar to the original one.
This means we can find matrices such that
%
\begin{equation}\label{eq:cccmats}
\begin{aligned}
  \Gammab\inv \Gammab^\mu \Gammab &= -\Gammab^\mu, &\quad
  \A  \Gammab^\mu \A\inv  &=  \Gammab^\mud, &\quad
  \At \Gammab^\mu \At\inv &= -\Gammab^\mud, \\ &&
  \B\inv  \Gammab^\mu \B  &=  \Gammab^\muc, &
  \Bt\inv \Gammab^\mu \Bt &= -\Gammab^\muc, \\ &&
  \C  \Gammab^\mu \C\inv  &=  \Gammab^\mut, &
  \Ct \Gammab^\mu \Ct\inv &= -\Gammab^\mut.
\end{aligned}
\end{equation}
%
Some of these matrices have names: $\Gammab$ is called the chirality matrix, either $\C$ or $\Ct$ is usually called the charge conjugation matrix, but that name would be more appropriate for $\B$ or $\Bt$.
Note that these matrices have only been defined up to an overall scalar multiple, as in \cref{eq:simsim}.
We will fix some of this ambiguity with normalisation conditions below.
We have also defined $\A,\At,\C,\Ct$ in the opposite way to $\Gammab,\B,\Bt$.
This is convenient due to the way that transposing reverses products.

When $d$ is odd, \cref{it:odd,it:oddchi} imply that $\Gammab$ does not exist and exactly one of each pair $(\A, \At)$, $(\B, \Bt)$, $(\C, \Ct)$, will exist.
The statements in the rest of this section should be prefaced with ``assuming that the matrices exist...''.

These transforms are all idempotent.
This means we can perform them twice to derive new identities:
%
\begin{equation*}
\begin{gathered}
\begin{alignedat}{5}
  \Gammab^\mu &= -(-\Gammab^\mu) &
        &= -(\Gammab\inv \Gammab^\mu \Gammab) &
        &= \Gammab\inv (-\Gammab^\mu) \Gammab \\ && &&
        &= \Gammab\invsq \Gammab^\mu \Gammab^2 &\quad
  &\implies &
  \Gammab^2 &= \gamma \I, \\
  \Gammab^\mu &= \Gammap[\mud]^\dag &
        &= (\A \Gammab^\mu \A\inv)^\dag &
        &= \A\invd \Gammab^\mud \A^\dag \\ && &&
%        &= \A\invd \A \Gammab^\mu \A\inv \A^\dag &
        &= (\A\inv \A^\dag)\inv \Gammab^\mu \A\inv \A^\dag &
  &\implies &
  \A\inv \A^\dag &= \alpha \I,
\end{alignedat} \\
\begin{aligned}
  \mathclap{\text{similarly:}} &&\qquad
  \B \B^\ast &= \beta \I, &\qquad
  \C\inv \C\trans &= \chi \I, \\
  \At\inv \At^\dag &= \alphat \I, &
  \Bt \Bt^\ast &= \betat \I, &
  \Ct\inv \Ct\trans &= \chit \I.
\end{aligned}
\end{gathered}
\end{equation*}
%
Wee are free to redefine these matrices as $\Gammab \rightarrow \lambda \Gammab, \A \rightarrow \lambda \A,$ etc.
Under these redefinitions the scalars above transform as
%
\begin{equation*}
  \gamma \rightarrow \lambda^2 \gamma, \qquad
  \alpha \rightarrow \frac{\lambda^\ast}{\lambda} \alpha, \qquad
  \beta \rightarrow \abs{\lambda}^2 \beta, \qquad
  \chi \rightarrow \chi,
\end{equation*}
%
with similar relations for the tilde versions.
These redefinitions allow us to impose some normalisation choices, noting that we can change the phase of $\alpha$ but not its magnitude, vice-versa for $\beta$, whereas $\chi$ cannot be changed at all.
%
\begin{equation}\label{eq:normscalar}
  \gamma = 1, \qquad
  \phase(\alpha) = 0, \qquad
  \abs{\beta} = 1,
\end{equation}
%
or in terms of the matrices
%
\begin{equation}\label{eq:normmats}
  \Gammab^2 = \I, \qquad
  \A\inv \A^\dag = \A \A\invd, \qquad
  \psib^\dag \A\inv \A^\dag \psib \geq 0 \;\; \forall \, \psib, \qquad
  \B^\ast \B^2 \B^\ast = \I.
\end{equation}
%
Again, the same relations apply to the tilde versions, except that in some situations it will be more convenient to choose to set \( \phase(\alphat) = \pi \).
These choices do not completely eliminate the scalar ambiguity.
For $\Gammab$ we can still use $\lambda = \pm 1$, for $\A$ any $\lambda \in \R$, for $\B$ any $\lambda$ with $\abs{\lambda} = 1$ and for $\C$ any $\lambda \in \Cb$.

The transformations in \cref{eq:cccmats} can also be performed by combining two others in either order.
We can use these combinations to find more scalars and normalisation conditions.
First, for $\A,\B,\C$:
%
\begin{equation*}
\begin{alignedat}{5}
  \Gammab^\mud &= \Gammap[\mut]^\ast &
    &= \C^\ast \Gammab^\muc \C\invc &
    &= \C^\ast \B\inv \Gammab^\mu \B \C\invc &
    &\implies &
  \C^\ast \B\inv &= f \A, \\
   &= \Gammap[\muc]\trans &
    &= \B\trans \Gammab^\mut \B\invt &
    &= \B\trans \C \Gammab^\mu \C\inv \B\invt &
    &\implies &
  \B\trans \C &= \tilde{f} \A. \\
\end{alignedat}
\end{equation*}
%
We can relate these scalars to the earlier ones as follows
%
\begin{equation*}
\begin{alignedat}{2}
  \A\inv &= f \B \C\invc &
    &= \tilde{f} \C\inv \B\invt \\
  \A^\dag &= \frac{1}{f^\ast} \B\invd \C\trans &
    &= \frac{1}{\tilde{f}^\ast} \C^\dag \B^\ast.
\end{alignedat}
   \implies
\begin{alignedat}{3}
  \alpha \I &= \A\inv \A^\dag &
    &= \frac{\tilde{f}}{f^\ast} \C\inv \B\invt \B\invd \C\trans &
%      &= \frac{\tilde{f}}{f^\ast \beta} \C\inv \C\trans &
      &= \frac{\tilde{f} \chi}{f^\ast \beta} \I, \\ &&
    &= \frac{f}{\tilde{f}^\ast} \B \C\invc \C^\dag \B^\ast &
%      &= \frac{f \chi^\ast}{\tilde{f}^\ast} \B \B^\ast &
      &= \frac{f \beta \chi^\ast}{\tilde{f}^\ast} \I.
\end{alignedat}
\end{equation*}
%
So we have \( \frac{\tilde{f}}{f^\ast} = \frac{\alpha \beta}{\chi} \)
and \( \frac{f}{\tilde{f}^\ast} = \frac{\alpha}{\beta \chi^\ast} \).
If we take the complex-conjugate of the second equation and use the choices in \cref{eq:normscalar},
we find that \( \frac{\tilde{f}}{f^\ast} = \frac{f^\ast}{\tilde{f}} \),
or \( \tilde{f} = \pm f^\ast \) which implies that $\chi = \pm \alpha \beta$.
Furthermore, under the $\C \ra \lambda \C$ ambiguity \( f \rightarrow \lambda^\ast f\),
\(\tilde{f} \rightarrow \lambda \tilde{f} \).
By choosing $\lambda = 1/\tilde{f}$ we set $f = \pm1$ and $\tilde{f} = 1$.
%
\begin{equation}\label{eq:comboABC}
  \A = \B\trans \C = \pm \C^\ast \B\inv, \qquad
  \chi = \pm \alpha \beta.
\end{equation}
%
The first relation is a normalisation choice; the second is forced on us by \cref{eq:normscalar}.
This reduces the scalar ambiguity to simultaneous \( \A \rightarrow r \A \),
\( \B \rightarrow \e^{-\ir\phi} \B \), \( \C \rightarrow r\e^{\ir\phi} \C \)
with \( r, \phi \in \R \).
We can also derive corresponding relations with two of the three matrices/scalars replaced by the tilde versions, but without the freedom to scale $\C$ (we reserve the freedom to rescale $\Ct$ for \cref{eq:chiraltilde}).
%
\begin{equation}\label{eq:ABCtildes}
\begin{aligned}
%  \A &= \B\trans \C = \pm \C^\ast \B\inv, &
  f_1 \A &= \Bt\trans \Ct = \pm \Ct^\ast \Bt\inv, &\quad
  f_2 \At &= \B\trans \Ct = \pm \Ct^\ast \B\inv, &\quad
  f_3 \At &= \Bt\trans \C = \pm \C^\ast \Bt\inv, \\
%  {\chi} &= \pm {\alpha} {\beta}, &
  {\chit} &= \pm {\alpha} {\betat}, &
  {\chit} &= \pm {\alphat} {\beta}, &
  {\chi} &= \pm {\alphat} {\betat}.
\end{aligned}
\end{equation}
%
The signs in each column match.
The first row contains normalisation choices; the second row is forced on us by \cref{eq:normscalar}.
For odd $d$ only one of the four sets of relations from \cref{eq:comboABC,eq:ABCtildes} is possible.

When $d$ is even, the same approach leads to relations between tilde versions, tilde-free versions and $\Gammab$.
We find
%
\begin{equation}\label{eq:chiraltilde}
\begin{aligned}
  \A\inv \A^\dag = a \At\inv \At^\dag &= \alpha \I,    &
  \At &= \Gammab^\dag \A = a \A \Gammab, &
  \phase(\alpha) &= 0, &
  \abs{a} &= 1, \\
  \B \B^\ast = \pm \Bt \Bt^\ast &= \beta \I, &
  \Bt &= b \Gammab \B = \pm \frac{1}{b} \B \Gammab^\ast,  &
  \phase(b) &= 0, &
  \abs{\beta} &= 1, \\
  \C\inv \C\trans = \pm \Ct\inv \Ct\trans &= \chi \I,    &
  \Ct &= \Gammab\trans \C = \pm \C \Gammab. &
%  \A = \B\trans \C &= \pm \C^\ast \B\inv, &
%  \chi &= \pm \alpha \beta,
\end{aligned}
\end{equation}
%
The signs across each row match.
The first column is forced on us by other relations; the other columns contain normalisation choices.
The remaining scalar ambiguities are multiplying $\Gammab,\At,\Bt,\Ct$ by $\pm1$
and performing \( \At \rightarrow r \At \), \( \Bt \rightarrow \e^{-\ir\phi} \Bt \),
\( \Ct \rightarrow r\e^{\ir\phi} \Ct \) simultaneously with the transformations of $\A$, $\B$, $\C$ below \cref{eq:comboABC}.

For even $d$ we can derive more relations between these scalars and signs by substituting relations from the second column of \cref{eq:chiraltilde} into \cref{eq:ABCtildes}.
The overall result is
%
\begin{equation}\label{eq:constnorm}
\begin{gathered}
\begin{alignedat}{4}
  \A\inv \A^\dag &= s_1 \At\inv \At^\dag &
        &= \alpha \I, &
    \At &= \Gammab^\dag \A &
        &= s_1 \A \Gammab,
  \\
  \B \B^\ast &= s_2 \Bt \Bt^\ast &
        &= \beta \I, &
    \Bt &= \Gammab \B &
        &= s_2 \B \Gammab^\ast,
  \\
  \C\inv \C\trans &= s_3 \Ct\inv \Ct\trans &
        &= \hat{s}\, \alpha \beta \I, &\qquad
    \Ct &= \Gammab\trans \C &
        &= s_3 \C \Gammab,
  \\
\end{alignedat}
\\
\begin{alignedat}{3}
  \A &= \B\trans \C &
        &= \hat{s}\, \C^\ast \B\inv, &\qquad
    \Gammab^2 &= \I,
  \\
  \A &= \Bt\trans \Ct &
        &= s_2 s_3 \hat{s}\, \Ct^\ast \Bt\inv, &\qquad
    \A\inv \A^\dag &= \A \A\invd,
  \\
  s_3 \At &= \B\trans \Ct &
        &= s_1 s_3 \hat{s}\, \Ct^\ast \B\inv, &\qquad
    \psib^\dag \A\inv \A^\dag \psib &\geq 0 \;\; \forall \, \psib,
  \\
  s_1 s_2 \At &= \Bt\trans \C &
        &= s_1 s_2 \hat{s}\, \C^\ast \Bt\inv, &
    \B^\ast \B^2 \B^\ast &= \I,
\end{alignedat}
\\
  \text{or}\quad
  \phase(\alpha) = 0, \qquad
  \abs{\beta} = 1, \qquad
  s_i^2 = \hat{s}^2 = 1.
  %s s' s'' &= 1.
\end{gathered}
\end{equation}
%
The first column and last row are forced on us by \cref{eq:normscalar} and other relations; the second column contains normalisation choices.
For odd $d$, on each of the first three rows only one of the two left equations, and none of the right, will exist.
Only one of the left sides of the next four rows will exist.

In total, we have four undetermined signs $s_i$, $\hat{s}$, and one undetermined complex number $\alpha \beta$.
For odd $d$, we can absorb the $s_i$ into $\alpha$, $\beta$ or $\hat{s}$.

The remaining scalar ambiguities are multiplying $\Gammab,\At,\Bt,\Ct$ by $\pm1$ (for even $d$ only)
and simultaneous \( \A \rightarrow r \A \),
\( \B \rightarrow \e^{-\ir\phi} \B \), \( \C \rightarrow r\e^{\ir\phi} \C \)
\( \At \rightarrow r \At \), \( \Bt \rightarrow \e^{-\ir\phi} \Bt \),
\( \Ct \rightarrow r\e^{\ir\phi} \Ct \) with \( r, \phi \in \R \) (for odd $d$ only half of these will exist).


%-------Section------------------------------------------------------------------

\section{Change of basis}\label{sec:basis}

Suppose we make a change of basis for our spinors, described by a similarity transform:
%
\begin{equation*}
  \Gammab'^{\mu} = \S\inv \Gammab^\mu \S.
\end{equation*}
%
This new representation will have its own set of chirality/charge conjugation matrices, as in \cref{eq:cccmats}.
How are they related to the corresponding matrices for the previous basis?

Lets illustrate one of these calculations
%
\begin{equation*}
\begin{gathered}
\begin{aligned}
  \Gammab'^\mud 
    &= (\S\inv \Gammab^\mu \S)^\dag 
    = \S^\dag \Gamma^\mud \S\invd 
    = \S^\dag \A \Gammab^\mu \A\inv \S\invd 
    = \S^\dag \A \S \Gammab^\mu \S\inv \A\inv \S\invd \\
    &= (\S^\dag \A \S) \Gammab'^\mu (\S^\dag \A \S)\inv,
    \qquad \implies \quad 
    \A' \propto \S^\dag \A \S, 
\end{aligned}
\\
\begin{aligned}
  \psib^\dag (\S^\dag \A \S)\inv (\S^\dag \A \S)^\dag \psib 
    &= \psib^\dag (\S^\dag \A \S)\inv (\S^\dag \A \S)^\dag \psib  
     = \psib^\dag \S\inv \A\inv \S\invd \S^\dag \A^\dag \S \psib \\
    &= \psib^\dag \S\inv \A\inv \A^\dag \S \psib 
     = \alpha\, \psib^\dag \S\inv \S \psib 
     = \alpha\, \psib^\dag \psib 
     \geq 0.
\end{aligned}
\end{gathered}
\end{equation*}
%
Once we apply our normalisation condition we find, up to a positive real scalar multiple, 
\( \A' = \S^\dag \A \S \).
Similar arguments give us
%
\begin{equation}\label{eq:ccctransform}
\begin{aligned}
  \Gammab' &= \S\inv \Gammab \S, &
  \A' &= \S^\dag \A \S, &
  \B' &= \S\inv \B \S^\ast, &
  \C' &= \S\trans \C \S, 
\\ &&
  \At' &= \S^\dag \At \S, &
  \Bt' &= \S\inv \Bt \S^\ast, &
  \Ct' &= \S\trans \Ct \S.
\end{aligned}
\end{equation}
%
We still have the same ambiguities as at the end of the \hyperref[eq:constnorm]{last section}, but it makes sense to choose \( r \e^{\ir\phi} = 1 \).
Regardless, we can substitute \cref{eq:ccctransform} into \cref{eq:constnorm} to compute the scalars:
%
\begin{equation*}
  \alpha' = \alpha, \qquad
  \beta' = \beta, \qquad
  s' = s, \qquad
  \hat{s}' = \hat{s}.
\end{equation*}
%
In other words, these scalars are invariant under a change of basis and are constant throughout any similarity class of representations.

Although there are two similarity classes of irreps when $d$ is odd, the same set of $\A/\At$, $\B/\Bt$, $\C/\Ct$ will work for both $\brc{\Gammab^\mu}$ and $\brc{ -\Gammab^\mu }$.
Therefore \cref{it:oddchi} implies that both classes have the same $\alpha, \beta, s, \hat{s}$.

\Cref{it:even,it:odd} imply that the only thing these constants can depend on is the space-time dimensionality, $p$ and $q$.
In the \hyperref[sec:explicit]{next section} we will compute these quantities for one specific representation that we will construct.
The same constants apply to all representations of that dimensionality.

But before that, how should a spinor transform under this change of basis?
The important thing is that the Dirac equation is covariant:
%
\begin{equation*}
  \ir \Gammab^\mu \partial_\mu \psib = m \psib
  \qquad \means \qquad
  \ir \Gammab'^\mu \partial_\mu \psib' = m \psib'.  
\end{equation*}
%
We can construct $\psib'$ as follows
%
\begin{equation*}
\begin{aligned}
  \ir \Gammab^\mu \partial_\mu \psib &= m \psib &
  &\implies &
  \ir \prn{\S \Gammab'^\mu \S\inv} \partial_\mu \psib &= m \psib \\ &&
  &\implies &
  \ir \Gammab'^\mu \partial_\mu \prn{\S\inv\psib} &= m \prn{\S\inv\psib} &
  &\implies &
  \psib' &= \S\inv \psib.
\end{aligned}
\end{equation*}
%
This means we have the following transformations:
%
\begin{equation}\label{eq:spinorbasis}
  \psib' = \S\inv \psib, \qquad
  \psib'^\dag \A' = \psib^\dag \A \S, \qquad
  \B' \psib'^\ast = \S\inv \B \psib^\ast, \qquad
  \psib'\transp \C' = \psib\trans \C \S,
\end{equation}
%
and similar for tilde versions.
Note that $\psib^\dag \psib$ is not an invariant quantity.
However, with the transformations above, we can construct these invariants:
%
\begin{equation}\label{eq:spinorinvt}
\begin{gathered}
  \psib^\dag \A \psib, \qquad
  \psib^\dag \A \B \psib^\ast, \qquad
  \psib\trans \C \psib, \qquad
  \psib\trans \C \B \psib^\ast, \qquad
  \psib^\dag \A \Gammab^\mu \psib, \\
  \psib^\dag \A \Gammab^\mu \Gammab^\nu \cdots \psib, \qquad
  \psib^\dag \A \Gammab^\mu \Gammab^\nu \cdots \B \psib^\ast, \qquad
  \psib\trans \C \Gammab^\mu \Gammab^\nu \cdots \Gammab \psib, \qquad
  \ldots
\end{gathered}
\end{equation}
%
The quantity \( \psib^\dag \A \) occurs frequently enough to have its own notation \( \overline{\psib} \equiv \psib^\dag \A \).


%-------Section------------------------------------------------------------------

\section{One explicit representation}\label{sec:explicit}


 



%================================================================================
\bibliographystyle{utcaps_sl}
\bibliography{journals,qft,gr,maths}
%================================================================================
\end{document}

