% ----------------------------------------------------------------
% Article Class (This is a LaTeX2e document)  ********************
% ----------------------------------------------------------------
\documentclass[12pt]{article}
\input{sl_preamble.tex}
\input{sl_definitions.tex}
\input{sl_symbols.tex}
% ----------------------------------------------------------------
\newcommand{\inv}{^{-1}}
\newcommand{\invsq}{^{-2}}
% ----------------------------------------------------------------

\begin{document}

\title{Extrinsic curvature for higher codimension}%
\author{Subhaneil Lahiri}%
%\address{Stanford University}%
%\thanks{}%
%\date{}%
% ----------------------------------------------------------------
\maketitle
\begin{abstract}
  We define scalar measures of local extrinsic curvature and scale invariant measures of global extrinsic curvature for submanifolds of $\R^n$.
\end{abstract}
% ----------------------------------------------------------------

\section{Fundamental forms (abstract)}\label{sec:fundforms}

Let $\CM$ be a submanifold of $\R^n$m and $m_1,m_2$ are points in $\CM$.
The Grassmannian ``distance'' between these two points, $D_{III}(m_1,m_2)$, is (the sine of?) the largest principal angle between the tangent planes $T_{m_1}\CM$ and $T_{m_2}\CM$, regarded as subspaces of $T_{m_i}\R^n = \R^n$.
This is a nonlocal measure of extrinsic curvature.
It is related to a local measure of extrinsic curvature, the third fundamental form \cite{clarkson2008tighter}:
%
\begin{equation}\label{eq:grassmannian_dist}
  D_{III}(m_1,m_2) = \int \!\!\! \sqrt{K(u(t),u(t))} \, \dr t,
\end{equation}
%
where $u(t)$ is the tangent vector to the ``shortest'' path $m(t)$ from $m_1$ to $m_2$.

The first fundamental form of $\CM$ is the induced metric:
%
\begin{equation}\label{eq:first_fund}
  h(u,v) = \av{u,v}_\CM = \av{U,V}_{\R^n},
\end{equation}
%
where $u,v$ are tangent vectors to $\CM$, $\av{\;,\;}_\CM$ is the inner product on $T_{m}\CM$, $U,V$ are the push-forwards of $u,v$ to $\R^n$ and $\av{\;,\;}_{\R^n}$ is the inner product on $\R^n$.
The first fundamental form maps two tangent vectors of $\CM$ to a scalar.

The (vector valued) second fundamental form and shape operators are defined as \cite{gallot1990riemannian}
%
\begin{equation}\label{eq:second fund}
  S(u,v) = \widetilde{D}_U V - D_U V,
  \qquad \av{S(u),v}_\CM = S(u,v), 
\end{equation}
%
where $\widetilde{D}$ and $D$ are the covariant derivatives on $\R^n$ and $\CM$ respectively.
 Wehave used the same letter for the second fundamental form and the shape operators to conserve alphabetical resources, they can be distinguished by the number of arguments.
The second fundamental form maps two tangent vectors of $\CM$ to a tangent vector of $\R^n$.
The shape operator maps one tangent vectors of $\CM$ to an object that is both a tangent vectors of $\CM$ and a tangent vector of $\R^n$.

The third fundamental form is defined by \cite{Eschenburg2010228}
%
\begin{equation}\label{eq:third fund}
  K(u,v) =  \av{ \av{S(u),S(v)}_\CM }_{\R^n}.
\end{equation}
%
It maps two tangent vectors of $\CM$ to a scalar.


\section{Constructing extrisic curvatures (concrete)}\label{sec:scalarcurv}

Let $x^i$, $i=1 \ldots n$, be cartesian coordinates on $\R^n$.
Let $\sigma^\alpha$ , $\alpha=1 \dots d$, be coordinates on $\CM$.%
\footnote{For us, $\sigma$ parameterise translations of the image and $x$ could represent pixel values or neuron outputs.}
As $\CM$ is not necessarily Euclidean, we will have to distinguish covariant indices (subscripts) from contravariant indices (superscripts) for Greek letters.
We need not make this distinction for roman letters, but we will use superscripts to keep them out of the way.
We also use the Einstein summation convention, where there is implicit summation over indices that appear twice on the same side of any equation.

We parameterise $\CM$ as
%
\begin{equation}\label{eq:embedding}
  \begin{aligned}
  x^i &= \phi^i(\sigma), &
    \qquad &\text{with} \qquad &
      \phi^i_\alpha &= \pdiff{\phi^i}{\sigma^\alpha}, \\ 
    &&&\text{and}&
      \phi^i_{\alpha\beta} &=  \pdiff{{}^2\phi^i}{\sigma^\alpha\p\sigma^\beta}.
  \end{aligned}
\end{equation}
%
We can use $\phi^i_\alpha$ to push-forward vectors and pull back forms.

We denote the units of $x$ and $\sigma$ by $[x]$ and $[\sigma]$ respectively.
Then
%
\begin{equation}\label{eq:phiunits}
  \brk{\phi^i} = [x],
  \qquad
  \brk{\phi^i_\alpha} = [x][\sigma]\inv,
  \qquad
  \brk{\phi^i_{\alpha\beta}} = [x][\sigma]\invsq.
\end{equation}
%
This tells us how these quantities would scale if we were to rescale $x$ or $\sigma$.

The induced metric and its matrix-inverse are:
%
\begin{equation}\label{eq:inducedmet}
\begin{aligned}
  h_{\alpha\beta} &= \phi^i_\alpha \phi^i_\beta, \qquad&
  h^{\alpha\beta} &= \prn{h_{\alpha\beta}}\inv, \\
  \brk{h_{\alpha\beta}} &= [x]^2[\sigma]\invsq, \qquad&
  \brk{h^{\alpha\beta}} &= [x]\invsq[\sigma]^2. \\
\end{aligned}
\end{equation}
%
The inverse metric can be pushed-forward to $\R^n$:
%
\begin{equation}\label{eq:proj}
  h^{ij} = \phi^i_\alpha \phi^j_\beta \, h^{\alpha\beta}, \qquad
  \tilde{h}^{ij} = \delta^{ij} - h^{ij}, \qquad
  \brk{h^{ij}} = \brk{ \tilde{h}^{ij}} = 1.
\end{equation}
%
These are the orthogonal projection operators parallel and perpendicular to $T_m\CM$ respectively.

The second fundamental form is given by:
%
\begin{equation}\label{eq:extcurvtensor}
  S^i_{\alpha\beta} = \phi^i_{\alpha\beta} - \Gamma^\gamma_{\alpha\beta} \, \phi^i_\gamma
                    = \tilde{h}^{ij} \, \phi^j_{\alpha\beta},
  \qquad
  \brk{S^i_{\alpha\beta}} = [x][\sigma]\invsq,
\end{equation}
%
where $\Gamma^\gamma_{\alpha\beta}$ are the Christoffel symbols for the covariant derivative on $|CM$.

The third fundamental form is given by:
%
\begin{equation}\label{eq:grassmet}
  K_{\alpha\beta} = h^{\gamma\delta} \,  S^i_{\alpha\gamma} S^i_{\beta\delta}
                  = h^{\gamma\delta} \, \tilde{h}^{ij} \, \phi^i_{\alpha\gamma} \phi^j_{\beta\delta}
  \qquad
  \brk{K_{\alpha\beta}}=[\sigma]\invsq.
\end{equation}
%
If we use the special coordinate system of \cite[Appendix B]{clarkson2008tighter}, we get
%
\begin{equation}\label{eq:specialcoord}
  \sigma^\alpha = x^\alpha, \quad \alpha=1\ldots d,
  \qquad \implies \qquad
  K_{\alpha\beta} = \sum_{i=d+1}^n \sum_{\gamma=1}^d \phi^i_{\alpha\gamma} \phi^i_{\beta\gamma},
\end{equation}
%
in agreement with the expression therein.
The Grassmanian distance between two points is
%
\begin{equation}\label{eq:grass_dist_explicit}
  D_{III}(\sigma_1,\sigma_2) = \int\!\!\! \sqrt{K_{\alpha\beta} \, \dot{\sigma}^\alpha(t)\dot{\sigma}^\beta(t)}\, \dr t,
\end{equation}
%
where $\sigma(t)$ is the ``shortest'' path connecting $\sigma_1$ and $\sigma_2$

If we think of $K_{\alpha\beta}$ as a linear operator, it maps vectors to covectors.
Therefore, it does not make sense to talk of its eigenvalues, as they would be coordinate dependent quantities.
Instead, consider
%
\begin{equation}\label{eq:curvop}
\begin{aligned}
  K^\alpha\dn{\beta} &= h^{\alpha\gamma}K_{\gamma\beta},  &
  K &= K^\alpha\dn{\alpha}, \\
  K^\alpha\dn{\beta} \, u^\beta_{(a)} &= \lambda^{(a)}_K \, u^\alpha_{(a)}, \qquad &
  \brk{K^\alpha\dn{\beta}} &= \brk{K} = \brk{\lambda^{(a)}_K} = [x]\invsq.
\end{aligned}
\end{equation}
%
Under a coordinate change, $K^\alpha\dn{\beta}$ undergoes a similarity transform.
Therefore, the eigenvalues, $\lambda^{(a)}_K$, are scalars.

we can use the square roots of the determinants of the first and third fundamental forms to construct invariant measures for integration over $\CM$.
They have units:
%
\begin{equation}\label{eq:measure units}
  \brk{\sqrt{\det h_{\alpha\beta}} \, \dr^d \sigma} = [x]^d,
  \qquad
  \brk{\sqrt{\det K_{\alpha\beta}} \, \dr^d \sigma} = 1.
\end{equation}
%
We can then construct the following global,, coordinate-invariant, dimensionless measures of extrinsic curvature:
%
\begin{equation}\label{eq:globalextr}
\begin{aligned}
  K_{(a)} &= \int_\CM \prn{\lambda^{(a)}_K}^{d/2} \sqrt{\det h_{\alpha\beta}} \, \dr^d \sigma, \\
  K_{\tr} &= \int_\CM K^{d/2} \sqrt{\det h_{\alpha\beta}} \, \dr^d \sigma
         = \int_\CM \prn{\sum_a \lambda^{(a)}_K}^{d/2} \sqrt{\det h_{\alpha\beta}} \, \dr^d \sigma, \\
  K_{\det} &= \int_\CM \sqrt{\det K_{\alpha\beta}} \, \dr^d \sigma
         = \int_\CM \prn{\prod_a \lambda^{(a)}_K}^{1/2} \sqrt{\det h_{\alpha\beta}} \, \dr^d \sigma. \\
\end{aligned}
\end{equation}
%
For our case of $d=2$, the second line is the sum of the first line over $a$.
For higher dimensional manifolds, one could come up with many other ways of combining eigenvalues in the integral.

%%%%%%%%%%%%%%%%%%%%%%%%%%%%%%%%%%%%%%%%%%%%%%%%%%%%%%%%%%%%%%%%%%%%%%%%%%

\bibliographystyle{utcaps_sl}
\bibliography{maths}

\end{document}
% ----------------------------------------------------------------
