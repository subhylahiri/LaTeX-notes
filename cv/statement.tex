% -*- TeX -*- -*- US -*-
% ----------------------------------------------------------------
% arXiv Paper ************************************************
%
% Subhaneil Lahiri's template
%
% Before submitting:
%    Comment out hyperref
%    Comment out showkeys
%    Replace \input{mydefs.tex} with its contents
%       or include mydefs.tex in zip/tar file
%    Replace \input{newsymb.tex} with its contents
%       or include newsymb.tex in zip/tar file
%    Put this file, the .bbl file, any picture or
%       other additional files and natbib.sty
%       file in a zip/tar file
%
% **** -----------------------------------------------------------
\documentclass[12pt]{article}
% Preamble:
\usepackage{a4wide}
\usepackage[centertags]{amsmath}
\usepackage{amssymb}
%\usepackage{amsthm}
\usepackage[sort&compress,numbers]{natbib}
%\usepackage{citeB}
\usepackage{ifpdf}
%\usepackage{graphicx}
%\usepackage{graphics} for finding documentation only
%\usepackage{xcolor}
%\usepackage{pgf}
\ifpdf
\usepackage[pdftex,bookmarks]{hyperref}
\else
\usepackage[hypertex]{hyperref}
%\DeclareGraphicsRule{.png}{eps}{.bb}{}
\fi
%
% ----------------------------------------------------------------
\vfuzz2pt % Don't report over-full v-boxes if over-edge is small
\hfuzz2pt % Don't report over-full h-boxes if over-edge is small
%\numberwithin{equation}{section}
%\renewcommand{\baselinestretch}{1.5}
% ----------------------------------------------------------------
% New commands etc.
\input{mydefs.tex}
\input{newsymb.tex}
%
%%%%%%%%%%%%%%%%%%%%%%%%%%%%%%%%%%%%%%%%%%%%%%%%%%%%%%%%%%%%%%%%%%%%%%%%%%
% Title info:
\title{Research statement}
%
% Author List:
%
\author{Subhaneil Lahiri}

\begin{document}

\maketitle


%%%%%%%%%%%%%%%%%%%%%%%%%%%%%%%%%%%%%%%%%%%%%%%%%%%%%%%%%%%%%%%%%%%%%%%%%%



%%%%%%%%%%%%%%%%%%%%%%%%%%%%%%%%%%%%%%%%%%%%%%%%%%%%%%%%%%%%%%%%%%%%%%%%%%
% Beginning of Article:
%%%%%%%%%%%%%%%%%%%%%%%%%%%%%%%%%%%%%%%%%%%%%%%%%%%%%%%%%%%%%%%%%%%%%%%%%%


I would like to perform research in computational neuroscience.

\subsection*{Graduate research}

My graduate work was based on the much celebrated AdS/CFT correspondence that relates certain theories of gravity to certain lower dimensional non-gravitational theories, particularly in the use of the non-gravitational theory to study black holes.

The early part of my graduate work was concerned with the thermodynamics of black holes with extra symmetries that could render comparisons between the two theories tractable. In \cite{Biswas:2006tj}, we studied a sector with greater symmetry and were able to demonstrate that the statistical mechanics of the two theories match exactly, suggesting that the this approach could work when applied to the black hole.

My later work focused in the regime where the non-gravitational theory is well approximated by fluid mechanics. The common approach is to use the black holes to calculate properties of the fluid. In \cite{Lahiri:2007ae,Bhattacharyya:2007vs,Bhattacharya:2009gm}, we turned this around and used the fluid to study the black holes. This approach uses a branch of physics that is several centuries old, in which we have developed a great deal of intuition from theory as well as everyday experience, to provide insight into aspects of black hole physics that are difficult to study directly.


\subsection*{Postdoctoral research}

My postdoctoral work has centered on the behavior of the \emph{Drosophila} larva. This is a particularly convenient model organism as it has a relatively small number of neurons, making it simpler than larger animals, there are many genetic tools to manipulate its nervous system and its body is transparent, allowing the use of optogenetics. It differs from other model organisms, such as \emph{C.\ elegans} in having a central nervous system and spiking neurons, making it more similar to larger, more complex organisms.

Navigation is a convenient type of behavior to study. The stimuli of interest, as well as the results of this behavior, can be characterized and controlled or measured effectively. Ultimately, one would like to characterize every part of the animal responsible for this, from the sensory neurons that receive the input, to the central nervous system that process the information and chooses the motion, to the muscles that produce the motor output.

In \cite{Lahiri2011}, we used fluorescent imaging of the muscle system in freely moving larvae to study the motor output in detail. We showed that the movements of the animals during navigation are composed of just two motor programs. To control navigation, the larval brain need only choose when, and how much, to use each program.

\subsection*{Future directions}

%\section*{Acknowledgements}



%%%%%%%%%%%%%%%%%%%%%%%%%%%%%%%%%%%%%%%%%%%%%%%%%%%%%%%%%%%%%%%%%%%%%%%%%%
%\section*{Appendices}
%\appendix
%%%%%%%%%%%%%%%%%%%%%%%%%%%%%%%%%%%%%%%%%%%%%%%%%%%%%%%%%%%%%%%%%%%%%%%%%%





%%%%%%%%%%%%%%%%%%%%%%%%%%%%%%%%%%%%%%%%%%%%%%%%%%%%%%%%%%%%%%%%%%%%%%%%%%

\bibliographystyle{utcaps_sl}
\bibliography{qft,string,adscft,gr,maths,larva}

\end{document}
